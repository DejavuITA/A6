\section{Introduzione}

Lo scopo di questa esperienza è quello di tarare una valvola a spillo (fine controlling needle valve) ricavando il flusso, anche se sarebbe meglio dire portata, in ingresso da misure di pressione della camera da vuoto, a seconda dell'apertura della valvola (misurata in numero di giri).

\section{Materiale utilizzato}

\begin{itemize}
	\item{sistema da vuoto costituito da una camera di volume $V = 5930 \pm 10$ \si{\centi\metre}$^3$, una pompa rotativa, una pompa ibrida turbomolecolare-molecular drag connessi in modo opportuno;}
	\item{valvola a spillo con regolazione micrometrica;}
	\item{multimetro e sistema di acquisizione dati;}
	\item{2 vacuometri Pirani;}
\end{itemize}
