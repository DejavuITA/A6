\section{Introduzione}

Lo scopo di questa esperienza è quello di tarare una valvola a spillo (\textit{fine controlling needle valve}) ricavando il flusso in ingresso per diversi valori di apertura della valvola (misurata in numero di giri). Ciò è stato ottenuto da misure di pressione della camera da vuoto. Facciamo notare che useremo la parola ``flusso'' anche se sarebbe più appropiato dire ``portata''.

\section{Materiale utilizzato}

\begin{itemize}
	\item{sistema da vuoto costituito da una camera di volume $V = 5930 \pm 10$ \si{\centi\metre}$^3$, una pompa rotativa, una pompa ibrida turbomolecolare-molecular drag connessi in modo opportuno;}
	\item{valvola a spillo con regolazione micrometrica;}
	\item{multimetro e sistema di acquisizione dati;}
	\item{2 vacuometri Pirani, uno in fondo alla pompa turbomolecolare e l'altro connesso alla camera da vuoto;}
\end{itemize}
