\section{Esecuzione}

La prima operazione da noi fatta è stata quella di mettere in moto l'impianto da vuoto portando la camera a una pressione di $10^{-3}$ \si{\pascal}, avendo cura di controllare la pressione raggiunta con il vacuometro a catodo freddo.

In seguito abbiamo misurato l'andamento della pressione interna della camera da vuoto in funzione del tempo con la valvola a spillo completamente chiusa. Ciò è necessario per misurare la variazione della pressione in camera che, una volta isolata dalla pompa turbomolecolare e dal resto dell'impianto, è dovuta ai soli effetti di degasamento.

Una volta eseguita questa prima serie di misure abbiamo riportato la camera ad una pressione interna di $10^{-3}$ \si{\pascal} per misurare il flusso in entrata a seconda dell'apertura della valvola. Abbiamo osservato la seguente procedura:
\begin{itemize}
	\item{portare la camera da vuoto\footnote{La valvola a spillo presenta al suo interno un volume morto di aria a pressione atmosferica, che se ignorato renderebbe scorrette le misure. Per ovviare a questo problema abbiamo chiuso la valvola con un tappo e un O-ring in viton e non con la manopola della stessa, in modo tale da rendere il volume morto parte della camera e quindi avere ovunque la stessa pressione misurata dal vacuometro a catodo freddo.} ad una pressione di circa $10^{-3}$ \si{\pascal}, controllando con il vacuometro a catodo freddo di essere ad una pressione minore del fondoscala del vacuometro Pirani;}
	\item{spegnere il vacuometro a catodo freddo per evitare l'usura o il deterioramento dello stesso a causa dell'aumento di pressione;}
	\item{isolare la camera da vuoto e aprire la valvola a spillo del numero di giri voluto;}
	\item{togliere il tappo posto a chiusura della valvola a spillo e misurare, tramite il multimetro e il sistema di acquisizione dati, il variare della pressione interna della camera in funzione del tempo;}
	\item{infine riportare la camera da vuoto ad una pressione dell'ordine $10^{-3}$ \si{\pascal}, isolando la pompa turbomolecolare e usando quella rotativa qualora necessario;}
\end{itemize}

Mentre la prima serie di misure è stata fatta con la valvola a spillo completamente chiusa, le misure successive hanno invece avuto lo scopo di ottenere i dati di flusso per un'apertura della valvola da 1 a 9 giri completi e di 5.2, 5.4, 5.6 e 5.8 giri.

Grazie alle condizioni meteo stabili, l'intera esperienza è stata svolta in laboratorio ad una temperatura ambiente di $XY \pm xy$ gradi centigradi, una pressione atmosferica di $941 \pm 0.5$ hPa ed una umidità del $55 \pm 0.5$ \%.

\section{Analisi dati}

Per ogni serie di dati abbiamo ricavato, con il metodo della regressione lineare, un valore della variazione di pressione $\frac{dP}{dt}$. Grazie a questi dati abbiamo poi potuto ottenere una stima del flusso in entrata grazie alla seguente relazione:

\begin{equation}
	Q_{tot} \,\, = \,\, V \frac{dP}{dt}
\end{equation}

Bisogna prestare attenzione che il flusso calcolato ($Q_{tot}$) grazie alla precedente equazione rappresenta il contributo di due differenti fattori: uno è dovuto alla perdita controllata di gas da parte della valvola a spillo ($Q_{valve}$), mentre il secondo è opera degli effetti di degasamento della camera da vuoto e delle altre parti dell'impianto ($Q_{deg}$).

Pertanto, per ricavare Q$_{valve}$ abbiamo dovuto sottrarre al flusso totale calcolato il flusso  dovuto alle perdite intrinseche e virtuali del sistema, ovvero:

\begin{equation}
	Q_{valve} \, = \, Q_{tot} \, - \, Q_{deg} \, = \, V \left[ \frac{dP}{dt} - \left(\frac{dP}{dt}\right)_{deg} \right]
\end{equation}

Dai dati sul flusso $Q_{valve}$ si può ricavare la conduttanza $C$ della valvola:

\begin{equation}
	Q_{valve} \, = \, C (P_{atm} - P) \quad \implies \quad C = \frac{Q_{valve}}{P_{atm} - P} = \frac{Q_{valve}}{P_{atm}}
\end{equation}
%
dove $P$ è la pressione all'interno della camera e $P_{atm}$ è la pressione atmosferica. Nell'ultimo passaggio si è supposto
che $P \ll P_{atm}$, approssimazione vera con buona precisione.

I risultati numerici sono esposti nella tabella sottostante, dove sono riportati i flussi e le conduttanze della valvola ottenute.
Nella prima riga è riportato il flusso di degasamento, e chiaramente non è presente la conduttanza.

\begin{table}
    \begin{tabular}{l c c c}
        \toprule
        Nr. giri & Q & $S\ped{rotativa}$ & $S\ped{turbo} \\
        \midrule
        1
        2
        3
        4
        5
        5.2
        5.4
        5.6
        5.8
        6
        7
        8
        9
        \bottomrule
    \end{tabular}
\end{table}




\begin{figure}[h!]
    \includegraphics[width=16cm]{graph.pdf}
    \caption{Il grafico mostra i valori elaborati del flusso in entrata a seconda dell'apertura della valvola a spillo. Le barre d'errore non sono
    mostrate in quanto comparabili con la dimensione dei punti sul grafico. L'asse y ha scala logaritmica}
    \label{fig:graph}
\end{figure}
