\section{Introduzione}

Lo scopo di questa esperienza di laboratorio è quella di approcciarci al montaggio di un impianto da vuoto che risulti in grado di raggiungere pressioni dell'ordine di $10^-3 \si{\pascal}$. Inoltre prenderemo dimestichezza con le varie component che costituiscono l'impianto da vuoto. Infine, se l'impianto risulterà funzionante, cercheremo di tarare un vacuometro Pirani.

\section{Materiale utilizzato}

Il materiale messo a nostra disposizione è il seguente:

\begin{itemize}
	\item{una camera in alluminio con un volume di $5930 \pm 10 cm^3$;}
	\item{una pompa rotativa con pressione di vuoto limite pari a circa $10^-2 \si{\pascal}$;}
	\item{flange e raccordi di varie dimensioni;}
	\item{2 valvole a membrana, 1 valvola a perdita calibrata, 1 valvola gate;}
	\item{2 vacuometri Pirani, 1 vacuometro a ionizzazione a catodo caldo e 1 vacuometo a ionizzazione a catodo freddo;}
	\item{un lettore AGC per i quattro sensori sopracitati;}
\end{itemize}

\section{Esecuzione dell'esperienza}

Per quanto riguarda il montaggio dell'impianto da vuoto abbiamo utilizzato il materiale sopraelencato, prestando attenzione che fossero rispettate le richieste dei tecnici di laboratorio, ovvero che l'impianto soddisfacesse i seguenti requisiti:

\begin{itemize}
	\item{si doveva poter isolare la pompa turbomolecolare dal resto dell'impanto;}
	\item{si doveva avere la possibilità di riportare la camera da vuoto alla pressione atmosferica senza spegnere la pompa turbomolecolare, che lavora correttamente a pressioni inferiori di circa $10^-2 \si{\pascal}$;}
\end{itemize}

Per non tediare inutilmente i lettori con la descrizione del montaggio dell'impianto da vuoto passo per passo preferiamo riportare di seguito uno schizzo del risultato finale da noi ottenuto. Figura ???

Invece per la taratura del vacuometro Pirani abbiamo portato il sensore oltre il range inferiore di lettura. Successivamente abbiamo tarato i due sensori presenti sullo stesso, identificati con A (atmosfera) e V (vuoto), affinche la tensione generata dallo strumento sia di 10 V a pressione atmosferica e di 2 V alla pressione limite inferiore dello strumento.   