\section{Introduzione}

L'esperienza consiste nella misura della conduttanza di quattro tubi di diverse lunghezze e diametri.
La misura si ottiene, mediante due misure di pressione, grazie alla definizione di conduttanza:

\begin{equation}
    Q = C \cdot \Delta P
\end{equation}

I valori di conduttanza misurati verranno poi confrontati con i valori calcolati teoricamente.
Facciamo presente sin da subito che le misure sono state effettuate in regimi di flusso
differenti.

\section{Materiale utilizzato}

\begin{itemize}
    \item{Sistema da vuoto composto da una pompa rotativa, due vacuometri pirani (già tarati) e alcuni tubi di raccordo.}
    \item{Valvola a spillo a regolazione micrometrica (tarata nelle esperienze precedenti).}
    \item{Tubi in esame (lunghezza x diametro in millimetri): 8000x4, 800x4, 8000x2.5, 800x2.5}
    \item{Sistema di acquisizione dati}
\end{itemize}
