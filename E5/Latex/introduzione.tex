\section{Introduzione}

Lo scopo di questa esperienza di laboratorio è quello di misurare la velocità di pompaggio delle due pompe, rotativa e turbomolecolare, in nostro possesso. Vogliamo ricavare il volume di gas che la pompa riesce ad evacuare dall'impianto da vuoto per unità di tempo.

\section{Materiale utilizzato}

\begin{itemize}
	\item{Sistema da vuoto costituito da una camera di volume $V = 5930 \pm 10$ \si{\centi\metre}$^3$, una pompa rotativa e una pompa ibrida turbomolecolare-molecular drag connessi in modo opportuno;}
    \item{Valvola a spillo con regolazione micrometrica (tarata nell'esperienza precedente);}
	\item{Multimetro e sistema di acquisizione dati;}
	\item{2 vacuometri Pirani, uno connesso al fondo della turbo, l'altro alla camera;}
\end{itemize}
