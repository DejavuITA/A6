\section{Introduzione}

Lo scopo di questa esperienza di laboratorio è quello di calcolare la velocità di pompaggio delle pompe rotativa e turbomolecolare in nostro possesso. Quindi vogliamo ricavare il volume reale di gas in funzione del tempo che la pompa riesce ad evacuare dall'impianto da vuoto.

\section{Materiale utilizzato}

\begin{itemize}
	\item{sistema da vuoto costituito da una camera di volume $V = 5930 \pm 10$ \si{\centi\metre}$^3$, una pompa rotativa, una pompa ibrida turbomolecolare-molecular drag connessi in modo opportuno;}
	\item{valvola a spillo con regolazione micrometrica;}
	\item{multimetro e sistema di acquisizione dati;}
	\item{2 vacuometri Pirani;}
\end{itemize}
