\section{Grazie a misure di massa si ricava il tipo di gas presente in un dato volume}   

In questa seconda parte della relazione vogliamo riuscire a ricavare mediante due misure di massa, bottiglia piena di gas e bottiglia vuota di gas, e sfruttando la legge dei gas ideli il peso molare del gas e pertanto risalire al tipo di gas presente nella bottiglia.

\subsection{Acquisizione dei dati}

\begin{itemize}
	\item{Come prima operazione abbiamo cercato di svuotare al massimo delle nostre capacità la bottiglia sfruttando la pompa meccanica a membrana a doppio stadio, in modo da rimuovere la maggior quantità di gas possibile. Successivamente abbiamo pesato la bottiglia "vuota" e abbiamo registrato il suo valore;}
	\item{Di segueito abbiamo riempito la bottiglia di aria fino a raggiungere una pressione interna di circa 3 atmosfere ovvero $\,3\,10^3\,\,Pa$. Quindi abbiamo pesato nuovamente la bottiglia piena annotandone il valore;}
	\item{Successivamente abbiamo determinato grazie a misure di massa i vari tipi di gas contenuti all'interno della bottiglia nel seguente modo: si svuota la bottiglia mediante la pompa meccanica a membrana in modo da rimuovere il più possibile la presenza del precedente gas, si annota la masa della bottiglia "vuota" e successivamente la si riempie fino a raggiungere una pressione di circa $\,3\,10^3\,\,Pa$. Di seguito si pesa la bottigli piena del gas incognito e se ne annota il valore.}
\end{itemize}

\subsection{Analisi dei dati}

Sfruttando la legge dei gas ideali:

\begin{equation}
	P\,V \,=\, n\,R\,T
\end{equation}

dove P rappresenta la pressione del gas, V il volume del gas, R non è altro che la costante universale dei gas, T è la temperatura del gas e infine n è il numero di moli del gas, siamo riusciti a ricavare il numero di moli contenute nella bottiglia ad una data pressione di circa $\,3\,10^3\,\,Pa$.

Quindi abbiamo trovato che il numero di moli contenute nella bottiglia sono:

\begin{equation}
	n \,=\, \frac{P\,V}{R\,T} \,=\, tot moli
\end{equation}

Infine conoscendo il numero di moli contenute nella bottiglia grazie alle due misure di peso della bottiglia la prima relativa al peso della bottiglia "svuotata" dal gas, la seconda relativa al peso della bottiglia piena di gas, abbiamo ricavato la massa totale del gas che, divisa per il numero di moli contenute nella bottiglia, restituiva il valore della massa molare relativa al gas e questa permette quindi di identificare il gas contenuto nella bottiglia. In sintesi:

\begin{equation}
	m_[mol] \,=\, \frac{M}{n}
\end{equation}

Pertanto i risultati da noi ottenuti sono i seguenti:

\begin{itemize}
	\item{massa molare x}
	\item{massa molare y}
	\item{massa molare z}
\end{itemize}

