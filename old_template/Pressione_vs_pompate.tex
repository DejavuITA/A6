\section{Apparato Sperimentale}

Gli strumenti che abbiamo utilizzato per eseguire questo esperimento sono i seguenti:
\begin{itemize}
	\item{Una bottigla di vetro a cui è applicato un manometro Bourdon di risoluzione di lettura pari a 0.1 bar;}
	\item{Un sisiema di raccorderia utilizzato per collegare la pompa a vuoto con la bottiglia e il manometro a U di Torricelli con una risoluzione di lettura di 1 mmHg;}
	\item{Pompa a mano per raggiungere il basso vuoto;}
	\item{Pompa meccanica a membrana doppio stadio (vuoto limite 500 Pa);}
	\item{Cilindri graduati con le seguenti risoluzioni di misura: quello da 1 litro ha una risoluzione di 10 ml, quello da un quarto di litro ha una risoluzione pari a 5 ml ed infine il cilindro graduato da 100 ml ha una rsoluzione di 1 ml;}
	\item{Una bilancia manuale di risoluzione 0.1 g.}
\end{itemize}
%
Nota: %Ci teniamo a specificare che
le indicazioni sulla risoluzione degli strumenti non sono riportate nelle unità di misura del S.I. %sistema internazionale
in quanto abbiamo voluto lasciare le indicazione sulle modalità di misura dello strumento. Nel seguito della relazione tuttavia tutte le misure effettuate e i risultati trovati saranno espressi nelle opportune unità di misura dell'S.I.

\section{Andamento della pressione in funzione del tempo}

Ci proponiamo ora di analizzare l'andamento della pressione interna della bottiglia rispetto a una variabile discreta, che nel nostro caso è rappresentata dal numero di pompate effettuate.

\subsection{Acquisizione dei dati}

\begin{itemize}
	\item{Come prima operazione abbiamo ricavato il voume incognito della bottiglia riempiendola d'acqua distillata e misurando il volume del liquido contenuto travasandolo in cilindri graduati;}
	\item{Successivamente abbiamo valutato l'andamento della pressione interna alla bottiglia in funzione del numero di pompate effettuate;}
	\item{Di seguito abbiamo riempito la bottiglia di aria, grazie ad un compressore, fino a raggiungere una pressione interna di 3 atmosfere, equivalenti circa a $3 \cdot 10^5$ \si{\pascal}, e abbiamo valutato l'andamento della pressione interna alla bottiglia in funzione del tempo utilizzando la pompa meccanica a membrana a doppio stadio;}
\end{itemize}

\subsection{Analisi dei dati}

La stima fatta del volume della bottiglia è la seguente:

\begin{equation}
	\mathcal{V} \, = \, (2769 \pm 8) \; \si{\deci\meter}^3  
\end{equation}

tuttavia non siamo molto fiduciosi della sua accuratezza dal momento che durante le operazioni di travaso del liquido dalla bottiglia nei cilindri graduati una minima parte di questo è andata persa. Inoltre non siamo riusciti a stimare con precisione il volume occupato dal tappo della bottiglia che comprendeva anche il manometro di Bourdon.

Graficando i dati raccolti relativi all'andamento della pressione in funzione del numero di pompate effettuate otteniamo il grafico seguente, che come possiamo notare ha un andameto esponenziale. Quindi possiamo notare come avvicinandosi al valore del vuoto limite della pompa da noi utilizzata la quantità di gas estratta diminuisca drasticamente e pertanto l'abbassamento della pressione interna della bottiglia risulti sempre più impercettibile, infatti vi è un andamento asintotico della pressione attorno al valore di x Pa.

GRAFICO

%Per quanto riguarda l'andamento della pressione interna della bottiglia in funzione del tempo, quindi di una variabile che supponiamo continua, i dati da noi ricavati sono riportati nella seguente tabella:

%Il grafico dell'andamento della pressione in funzione del tempo trascorso dall'inizio del processo di svuotamento della bottiglia da parte della pompa meccanica a membrana a doppio stadio è riportato nel grafico sottostante, che come si può notare ha un andamento esponenziale come trovato sopra.

\subsection{Conclusioni parziali}

Non siamo riusciti purtroppo ad analizzare l'andamento della pressione in funzione del tempo, quindi di una variabile continua, a causa di una scarsa qualità del video realizzato e della mancanza nelle riprese della terminazione delle estremità del manometro di Torricelli, e pertanto del relativo anamento pressione vs tempo. Possiamo però supporre che la curva che avrebbe descritto la pressione in funzione del tempo sarebbe stata, come nel caso precedente, un'esponenziale, con un andamento asintotico verso pressioni tendenti al vuoto limite della pompa a membrana.
