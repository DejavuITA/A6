\section{Interferometro di Michelson}

Lo scopo in questa prima parte della relazione è quello di ricavare il valore dell'indice di rifrazione dell'aria e di un tipo di vetro.

\subsection{Materiale}

Il materiale a nostra disposizione è il seguente:
\begin{itemize}
	\item{Interferometro di Michelson premontato e preallineato;}
    \item{Camera da vuoto e pompa a membrana per la misura dell'indice di rifrazione dell'aria;}
    \item{Vetro di cui misurare l'indice di rifrazione;}
	\item{Valvola a spillo per la regolazione del flusso di aria in uscita dalla camera;}
	\item{Goniometro per modificare e misurare l'angolo di incidenza ($\theta_i$) della luce sul vetro. Il goniometro aveva una risoluzione di misura di 0.1 gradi;}
    \item{Laser di luce verde, di lunghezza d'onda $\lambda = \SI{543.5}{\nano\metre}$;}
\end{itemize}
Ricordiamo che il laser è una sorgente di luce monocromatica e coerente, inoltre i raggi risultano essere collimati ovvero paralleli tra di loro.


