\section{Introduzione}

Lo scopo di questa esperienza di laboratorio è quello di misurare la pressione
di vapore dell'acqua all'equilibrio nell'intervallo di temperatura tra i 35$^\circ$C
e 85$^\circ$C. Quindi, grazie alle misure di pressione e temperatura effettuate, verificheremo
l'equazione di Clausius-Claperyon, che descrive la variazione della pressione di vapore
in funzione della temperatura lungo la curva di equilibrio tra due fasi di una stessa sostanza,
che nel nostro caso sono lo stato liquido e quello gassoso.
Infine, sfruttando i dati raccolti e la legge sopracitata cercheremo di ricavare il valore del calore latente di vaporizzazione dell'acqua.