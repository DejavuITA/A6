\section{Intruduzione}

Lo scopo di questa esperienza di laboratorio è quella di misurare la pressione di vapore dell'acqua all'equilibrio nell'intervallo di temeratura tra i 48°C e 85°C. Quindi grazie alle misure di pressione vs temperatura potremmo verificare la bontà dell'equazione di Clausius-Claperyon, che descrive la variazione della pressione in funzione della temperatura lungo la curva di equilibrio tra due fasi di una stessa sostanza, che nel nostro caso sono lo stato liquido e quello gassoso.
Infine, sfruttando i dati raccolti che la legge sopracitata cercheremo di stimare un valore medio per l'entalpia di vaporizzazione dell'acqua.