<<<<<<< HEAD
\section{Introduzione}

L'obbiettivo di questa esperienza è verificare sperimentalmente l'equazione di Clapeyron, che indica
come varia la pressione di vapore al variare della temperatura. Per fare ciò si è misurata la pressione di vapore
a diverse temperature e si è verificato se l'andamento è compatibile con la legge di Clapeyron.

L'esperimeno permette anche la misura del calore latente di vaporizzazione dell'acqua, che è il secondo
scopo dell'esperimento.
=======
\section{Intruduzione}

Lo scopo di questa esperienza di laboratorio è quella di misurare la pressione di vapore dell'acqua all'equilibrio nell'intervallo di temeratura tra i 48°C e 85°C. Quindi grazie alle misure di pressione vs temperatura potremmo verificare la bontà dell'equazione di Clausius-Claperyon, che descrive la variazione della pressione in funzione della temperatura lungo la curva di equilibrio tra due fasi di una stessa sostanza, che nel nostro caso sono lo stato liquido e quello gassoso.
Infine, sfruttando i dati raccolti che la legge sopracitata cercheremo di stimare un valore medio per l'entalpia di vaporizzazione dell'acqua.
>>>>>>> b60442ae49fcd8ed0207361a91af82c82f0f82b5
