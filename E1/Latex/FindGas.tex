\section{Grazie a misure di massa si ricava il tipo di gas presente in un dato volume}   

In questa seconda parte della relazione vogliamo riuscire a ricavare mediante due misure di massa,
a bottiglia piena di gas e a bottiglia vuota, la massa molare di alcuni comuni gas. Per fare ciò
sfrutteremo la legge dei gas ideali. Uno dei gas era incognito e dovevamo riuscire a capire quale gas fosse.

\subsection{Acquisizione dei dati}

\begin{itemize}
	\item{Come prima operazione abbiamo ricavato il voume incognito della bottiglia riempiendola d'acqua distillata e misurando il volume del liquido contenuto travasandolo in cilindri graduati;}
	\item{Come prima operazione abbiamo cercato di svuotare al massimo delle nostre capacità la bottiglia sfruttando la pompa meccanica a membrana a doppio stadio, in modo da rimuovere la maggior quantità di gas possibile. Successivamente abbiamo pesato la bottiglia "vuota" e abbiamo registrato il suo valore;}
	\item{Di seguito abbiamo riempito la bottiglia di aria fino a raggiungere una pressione interna di circa $3 \cdot 10^3 \; \si{\pascal}$. Quindi abbiamo pesato nuovamente la bottiglia piena annotandone il peso;}
	\item{Il procedimento è stato ripetuto con tutti i gas, compreso quello incognito.}
\end{itemize}

\subsection{Analisi dei dati}

La stima fatta del volume della bottiglia è la seguente:

\begin{equation}
	\mathcal{V} \, = \, (2.769 \pm 0.008) \; \si{\deci\meter}^3  
\end{equation}
%
tuttavia non siamo molto fiduciosi della sua accuratezza dal momento che durante le operazioni di
travaso del liquido dalla bottiglia nei cilindri graduati una minima parte di questo è andata persa.
Inoltre non siamo riusciti a stimare con precisione il volume occupato dal tappo della bottiglia che
comprendeva anche il manometro di Bourdon. Questi ultimi contributi tendono tuttavia a cancellarsi
per cui non dovrebbero contribuire molto all'incertezza.

Sfruttando la legge dei gas ideali:

\begin{equation}
	P\,V \,=\, n\,R\,T
\end{equation}
%
si può ricavare il numero di moli di gas presente all'interno della bottiglia, ottenendo:

\begin{equation}
	n \,=\, \frac{P\,V}{R\,T}
\end{equation}

Quindi conoscendo il numero di moli contenute nella bottiglia, e conoscendo la massa del gas contenuto,
abbiamo calcolato la massa molare dividendo la massa per il numero di moli:

\begin{equation}
	M_{mol} \,=\, \frac{m}{n}
\end{equation}

Pertanto i risultati da noi ottenuti sono i seguenti:

\begin{itemize}
	\item{massa molare x}
	\item{massa molare y}
	\item{massa molare z}
\end{itemize}

