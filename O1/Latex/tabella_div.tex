\begin{table}[H]
    \centering
    \small
    \begin{tabular}{c c c c c c c c}
        \toprule
        $p_d \; [\si{\centi\metre}]$ & $D \; [\si{\centi\metre}]$ & $q_c \; [\si{\centi\metre}] $ &
        $x \; [\si{\centi\metre}]$ & $f_d \; [\si{\centi\metre}]$ & $h\ped{imm} \; [\si{\centi\metre}]$ & $m$\ped{h} & $m$\ped{pq} \\
        \midrule
		30.5 $\pm$ 0.1 & 23.1 $\pm$ 0.1 & 76.0 $\pm$ 0.1 & -12 $\pm$ 1  & -20 $\pm$ 3 & 0.79 $\pm$ 0.02 & 0.87 $\pm$ 0.02 & 0.9 $\pm$ 0.1 \\
		30.5 $\pm$ 0.1 & 25.2 $\pm$ 0.1 & 67.1 $\pm$ 0.1 & -12 $\pm$ 1  & -20 $\pm$ 4 & 0.64 $\pm$ 0.02 & 0.71 $\pm$ 0.02 & 0.7 $\pm$ 0.1 \\
		30.5 $\pm$ 0.1 & 27.2 $\pm$ 0.1 & 61.6 $\pm$ 0.1 & -12 $\pm$ 2  & -20 $\pm$ 4 & 0.55 $\pm$ 0.02 & 0.61 $\pm$ 0.02 & 0.6 $\pm$ 0.1 \\
        \bottomrule
    \end{tabular}
    \caption{Dati relativi alla lente divergente. Per quanto riguarda le incertezze, si legga la didascalia della Tabella \ref{tab:conv}.
    Gli ingrandimenti $m\ped{h}$ e $m\ped{pq}$ sono riportati in valor assoluto. L'immagine risultava rovesciata.}
    \label{tab:div}
\end{table}
