\section{Analisi}

Innanzitutto occorre ottenere il valore di intensità massima $I\ped{max}$ e $\vartheta\ped{max}$, poiché abbiamo misurato tre valori distinti.
Abbiamo quindi fatto la media dei valori registrati. Dopo questo primo passo ci siamo assicurati di esprimere tutti gli angoli rispetto
all'angolo $\vartheta\ped{max}$, ovvero lo abbiamo sottratto a tutti gli altri angoli.

Abbiamo graficato i dati, ottenendo lo spettacolare grafico in Figura XXX-ANAL-GATE-4-PORNO-GAY-XXX, tracciando una linea orrizzontale
che corrisponde al 5\% dell'intensità massima. Chiaramente la curva interseca due volte la linea del 5\%, poiché è simmetrica.
Si possono quindi trovare due valori diversi di apertura numerica, e noi li abbiamo calcolati entrambi, come segue.
Dal grafico abbiamo trovato i punti tra i quali passa la linea corrispondente al 5\%,
di cui indicheremo con $I\ped{sopra}$, $I\ped{sotto}$ e $\vartheta\ped{sopra}$, $\vartheta\ped{sotto}$ rispettivamente l'intensità e l'angolo.
A questo punto si può ottenere il valore di \emph{NA} interpolando linermente tra i due punti e imponendo:

\begin{equation}
    \begin{split}
        0.05 \cdot I\ped{max} &= I\ped{sopra} + (I\ped{sotto} - I\ped{sopra}) \cdot \frac{\emph{NA} - \vartheta\ped{sopra}}{\vartheta\ped{sotto} - \vartheta\ped{sopra}} \\
                              &= I\ped{sopra} + \Delta I \cdot \frac{\emph{NA} - \vartheta\ped{sopra}}{\Delta \vartheta}
    \end{split}
\end{equation}

Da cui si ha, banalmente:

\begin{equation}
    \emph{NA} = \frac{\Delta \vartheta}{\Delta I} (0.05 \cdot I\ped{max}) + \vartheta\ped{sopra}
\end{equation}

I valori che noi abbiamo calcolato sono i seguenti:


