\section{Analisi dati}

Come anticipato nell'introduzione in ottica la legge di Lambert-Beer, conosciuta anche come legge di Beer-Lambert-Bouguer, è una relazione empirica che correla la quantità di luce assorbita da un mezzo alla natura chimica, alla concentrazione ed allo spessore del mezzo attraversato.
Quando un fascio di luce (monocromatica) di intensità $I_0$ attraversa uno strato di spessore $l$ di un mezzo (una soluzione nel nostro caso), una parte di questa intensità viene assorbita dal mezzo stesso e una parte viene trasmessa con intensità residua $I_f$.
In questa breve descrizione del processo abbiamo omesso di considerare che una parte della radiazione luminosa subisce anche effetti di riflessione a causa della pareti del contenitore.

Quindi il rapporto tra le intensità della luce trasmessa ($I_f$) e incidente sul mezzo attraversato ($I_0$) è espresso dalla seguente relazione:

\begin{equation}
	I_f \,\,=\,\, I_0 \,\, e^{-kcL}
\end{equation}


\begin{figure}[hbtp]
        \centering
        \includegraphics[scale=0.4]{chem_dog.pdf}
\end{figure}