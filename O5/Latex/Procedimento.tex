\section{Procedimento}

\subsection{Preparazione}

Partendo da 2ml di soluzione uno molare di $CuSO_4$ a nostra disposizione abbiamo creato 4 soluzioni con diversa concentrazione (0.5, 0.25, 0.125, 0.0625) grazie alla seguente formula:
\begin{equation}\label{eq:proporzione}
	C_2 \,\, = \,\, C_1 \frac{V_1}{V_2}
\end{equation}

Per creare la prima soluzione abbiamo diluito i 2 ml fino ad arrivare a 4 ml di soluzione. In seguito ne abbiamo prelevato 2 ml e depositati in un contenitore trasparente che è stato a suo volta assicurato alla base in polistirolo. Con i successivi 2 ml abbiamo creato 4 ml di soluzione 0.25 molare, dei quali abbiamo prelevato 2 ml per l'esperimento. E così via anche per le altre due concentrazioni.


Infine ogni gruppo ha creato una soluzione con concentrazione molare tra 0.5 e 1. Il nostro gruppo aveva l'incarico di preparare la soluzione 0.75 molare.
Seguendo l'equazione (\ref{eq:proporzione}) abbiamo creato una soluzione di $CuSO_4$ 0.75 molare.

\begin{equation}
	V_2 \,\, = \,\, V_1 \frac{C_1}{C_2} \,\, = \,\, 1.5 \si{\milli\litre} \,\, \frac{1 \,\, \si{\mole\per\litre}}{0.75 \,\, \si{\mole\per\litre}} \,\, = \,\, 2 \,\, \si{\milli\litre}
\end{equation}

La soluzione ricavata è stata interamente versata nell'ultimo contenitore trasparente a nostra disposizione e il quale è stato infine riposto nella base di polistirolo

\subsection{Procedimento di misura}

Per ogni contenitore trasparente abbiamo misurato l'intensità del raggio di luce che lo attraversava.

\begin{figure}[hbtp]
        \centering
        \includegraphics[scale=0.43]{doge.pdf}
        \end{figure}