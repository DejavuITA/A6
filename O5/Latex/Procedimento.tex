
\section{Procedimento}

\subsection{Preparazione della soluzione}

Partendo da $2\,\,\si{\milli\litre}$ di soluzione uno molare di $CuSO_4$ a nostra disposizione abbiamo creato 4 soluzioni con diversa concentrazione (0.5, 0.25, 0.125, 0.0625) grazie alla seguente formula:

\begin{equation}
	C_f \,\, = \,\, C_i \,\, \frac{V_i}{V_f}
	\label{eq:proporzione}
\end{equation}
%
dove $C_f$ è la concentrazione molare finale della soluzione, $C_i$ è la concentrazione molare iniziale della soluzione, $V_i$ è il volume iniziale della soluzione mentre $V_f$ è il volume finale della soluzione.

In questa esperienza siamo partiti da una soluzione a concentrazione uno molare di $CuSO_4$ che mano mano abbiamo diluito per ottenerne nuove concentrazioni, inferiori all'uno molare naturalmente :-).
Pertanto abbiamo seguito questo semplice procediemnto per ottenere la serie di concentrazioni elencate sopra:

\begin{itemize}
	\item{siamo partiti con un volume di $2\,\,\si{\milli\litre}$ di soluzione di $CuSO_4$ alla concentrazione uno molare;}
	\item{abbiamo deciso di fare quattro diluizioni della sostanza iniziale. In particolare abbiamo dimezzato ogni volta la concentrazione precedente in modo da rendere la procedura operativa molto semplice;}
	\item{infatti dalla soluzione di partenza abbiamo creato la prima diluizione aggiungendo $2\,\,\si{\milli\litre}$ di acqua distillata. Abbiamo dunque portato il volume finale della soluzione a $4\,\,\si{\milli\litre}$;}
	\item{in questo modo abbiamo dimezzato la concentrazione di $CuSO_4$ iniziale;}
	\item{successivamente abbiamo mescolato la soluzione in modo da renderla omogenea. Abbiamo poi prelevato $2\,\,\si{\milli\litre}$ di sostanza che è stata versata in uno dei contenitori trasparenti;}
	\item{in seguito abbiamo ripetuto la procedura sopra descritta in modo da dimezzare ulteriormente la concentrazione di sostanza appena ottenuta. Si noti infatti che per le diluizioni successive la concentrazione di soluto non è più dell'uno molare, ma per la seconda diluizione è del 0.5, per la terza del 0.25 e così via;}
	\item{quindi alla fine abbiamo ottenuto quattro contenitori trasparenti con al loro interno $2\,\,\si{\milli\litre}$ di sostanza alle varie concentrazioni elencate sopra (0.5, 0.25, 0.125, 0.0625);}
\end{itemize}

Infine ogni gruppo ha creato una soluzione con concentrazione molare tra 1 e 0.5. Il nostro gruppo aveva l'incarico di preparare la soluzione 0.75 molare.
Quindi seguendo l'equazione (\ref{eq:proporzione}) abbiamo creato la soluzione di $CuSO_4$ con concentrazione 0.75 molare. Facciamo notare che in questo caso siamo partiti con $1.5\,\,\si{\milli\litre}$ di soluzione uno molare. 

\begin{equation*}
	V_f \,\, = \,\, V_i \,\, \frac{C_i}{C_f} \,\, = \,\, 1.5 \si{\milli\litre} \,\, \frac{1 \,\, \si{\mole\per\litre}}{0.75 \,\, \si{\mole\per\litre}} \,\, = \,\, 2 \,\, \si{\milli\litre}
\end{equation*}
%
In questo modo abbiamo ricavato quale doveva essere il volume finale di sostanza affinchè la sua concentrazione molare fosse del 0.75. Quindi abbiamo dovuto semplicemente aggiungere $0.5\,\,\si{\milli\litre}$ di acqua distillata alla soluzione iniziale.
La soluzione ricavata è stata interamente versata nell'ultimo contenitore trasparente a nostra disposizione che è stato infine riposto nella base di polistirolo.

\subsection{Misura dell'intensità luminosa finale}

Come prima operazione abbiamo misurato l'intensità iniziale del raggio laser ($I_0$) senza alcun ostacolo tra la sorgente e il misuratore di intensità. Abbiamo ottenuto il seguente valore:

\begin{equation*}
	I\ped{0\ped{1}} \,\,=\,\, 130 \pm 1 \,\,\, \si{\joule} \,\, ??
\end{equation*}
%
Quindi abbiamo misurato per ogni concentrazione di $CuSO_4$ l'intensità luminosa finale ($I_f$) del raggio laser. Abbiamo eseguito questa misura non solo sulle quattro concentrazioni da noi fatte ma anche su quelle comuni a tutti i gruppi, ovvero quelle comprese tra 1 e 0.5 molare. I valori da noi ottenuti sono riportati nella tabela seguente, Tabella \ref{tab:dati}.
Infine abbiamo misurato nuovamente il valore dell'intensità luminosa del raggio laser a vuoto. Abbiamo ottenuto un valore di:

\begin{equation*}
	I\ped{0\ped{2}} \,\,=\,\, 131 \pm 1 \,\,\, \si{\joule} \,\, ??
\end{equation*}
%
Questa seconda misura ci serve per accertarci che l'apparato durante tutte le misure sia rimasto, nei limiti del possibile, stabile. Infatti se le due intensità luminose fossero state troppo diferenti tra di loro si sarebbe dovuta ripetere l'intera procedura di misura. Per fortuna le due misure risultano compatibili entro le loro incertezze, quindi i dati ottenuti sono validi e possiamo procedere ad analiizzarli.

\begin{table}[t!]
    \centering
    \begin{tabular}{c c }
        \toprule
         Concentrazione molare $[\si{\mole\per\litre}]$ & Intensità $I_f \,\, [\si{\joule}]$ \\
        \midrule
		$1$ 	& $7 \pm 1$ \\
		$0.9$ 	& $10 \pm 1$ \\       
        $0.8$ 	& $13 \pm 1$ \\
        $0.75$ 	& $14 \pm 1$ \\
        $0.7$ 	& $17 \pm 1$ \\
        $0.65$ 	& $20 \pm 1$ \\
        $0.6$ 	& $22 \pm 1$ \\
        $0.5 \pm 0.014$ 	& $26 \pm 1$ \\
        $0.25 \pm 0.01$ 	& $49 \pm 1$ \\
        $0.125 \pm 0.006$ & $74 \pm 1$ \\
        $0.0625 \pm 0.003$ & $91 \pm 1$ \\
        \bottomrule
    \end{tabular}
    \caption{In questa tabella sono riportati sulla prima colonna i valori della concentrazione molare di solfato di rame ($CuSO_4$) che siamo andati ad analizzare man mano. Sulla seconda colonna sono riportati i valori dell'intensità luminosa ($I_f$) in uscita dalla provetta contenente la soluzione. Ricordiamo che la soluzione 1 molare è stata presa senza errore. E' imprtante notare che gli errori sulle concetrazioni non sono uniformi in quanto ogni concentrazione (a scendere) è stata ottenuta da quella precedente, pertanto l'inceretezza si propaga. Infine facciamo notare che l'incertezza è stata inserita soltanto per le concentrazioni ottenute dalla procedura descritta ad inizio paragrafo, mentre alle concentrazioni condivise dai vari gruppi si è deciso di non attribuire alcun'errore.}
    \label{tab:dati}
\end{table}
%
Inoltre abbiamo anche misurato l'intensità del raggio laser quando la soluzione era composta interamente da acqua distillata e abbiamo ottenuto il seguente valore:

\begin{equation}
	I\ped{H_2O} \,\,=\,\, 114 \pm 1 \,\,\, \si{\joule}
	\label{eq:acqua}
\end{equation}
%
Infine abbiamo misurato, col calibro (???), la larghezza del contenitore nel quale viene posta la soluzione. Il suo valore risulta essere il seguente:

\begin{equation*}
	L \,\,=\,\, XXX \pm XXX
\end{equation*}

\begin{figure}[hbtp]
        \centering
        \includegraphics[scale=0.43]{doge.pdf}
\end{figure}