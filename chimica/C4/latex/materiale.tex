\section*{Materiale}

\begin{itemize}
        \setlength{\parskip}{0pt}
        \item{Vetreria: beker per titolare, becker per il recupero delle soluzioni di lavaggio,
            buretta da 25 ml, imbuto, pipette, beuta da 50 ml per \ce{NaOH};}
        \item{pHmetro con relativo supporto per le due sonde: termometro ed elettrodo a vetro;}
        \item{Agitatore magnetico con ancoretta;}
        \item{\SI{100}{\milli\liter} di soluzione di ***;}% a concentrazione incognita;}
%        \item{Soluzione 1\% di fenolftaleina in etanolo, soluzione 0.1 M di \ce{NaOH} e acqua distillata;}
        \item{Francesco Pasa, Andrea Miani, Davide Bazzanella e tanta, tanta allegria.}
\end{itemize}

\section*{Procedura e incertezze}

Per prima cosa abbiamo calibrato il pHmetro con tre soluzioni tampone,
seguendo le istruzioni riportate nel manuale dello strumento.

Il principio che ci permette di titolare la soluzione è il seguente: l'acido acetico, essendo un acido debole,
si dissocia poco in acqua (circa il 4 \% a temperatura 25$^\circ$C),
per cui si trova quasi completamente nella forma \ce{CH3COOH}.
L'\ce{NaOH} è invece una base forte e quindi, avendo una costante $K\ped{B}$ molto alta,
si dissocia completamente. Aggiungendo l'idrossido di sodio alla soluzione di acido acetico,
avviene la seguente reazione

\begin{equation}
    \ce{CH3COOH\ped{(aq)} + OH- _{(aq)} -> CH3COO- _{(aq)} + H2O\ped{(l)}}
    \label{eq:reaz}
\end{equation}
%
(gli ioni \ce{Na+} sono spettatori e sono stati omessi dalla formula per rendere la lettura più semplice).

Questa reazione ha una costante di equilibrio $K\ped{eq} \simeq 10^9$. Si perviene a tale valore poichè l'acido acetico ha $K\ped{A} \simeq 10^{-5}$,
quindi l'acetato (\ce{CH3COO-}), che è la sua base coniugata, ha $K\ped{B} = K_w / K\ped{A} \simeq 10^{-9}$.
Essendo la eq. \eqref{eq:reaz} la reazione
inversa della reazione dell'acetato con l'acqua, si ha che $K\ped{eq} = 1/K\ped{B} \simeq 10^9$.

Poichè $K\ped{eq}$ è molto grande, la \eqref{eq:reaz} è una reazione spostata verso i prodotti. Questo significa che la reazione
è quasi stechiometrica; praticamente per ogni \ce{NaOH} aggiunto una molecola di acido acetico si dissocia producendo acetato
(ecco perchè abbiamo messo solo la freccia verso destra).

Si può sfruttare questo fatto nel seguente modo: si aggiunge gradualmente \ce{NaOH} alla soluzione titolanda misurandone allo
stesso tempo il pH. Graficando poi il pH in funzione della quantità di idrossido di sodio aggiunto si individua
il punto equivalente, ovvero il punto in cui $n(\ce{CH3COOH})_i = n(\ce{NaOH})$ (il pedice $i$ sta per iniziale),
facendo leva sul fatto che la reazione si può considerare stechiometrica. Il punto di equivalenza è
il flesso della curva di titolazione. Da questo punto in poi, tutto l'acido acetico è diventato acetato.

Nel nostro caso, avevamo $V_0 = 100$ ml di soluzione a concentrazione incognita di acido acetico. Abbiamo
considerato un errore tipo di risoluzione di 0.3 ml sul volume della soluzione di partenza. Abbiamo versato la soluzione in un
beker, versandoci poi anche il residuo di lavaggio del matraccio dove era contenuta. In questo modo abbiamo
trasferito quanto più \ce{CH3COOH} possibile nel beker. Abbiamo quindi aggiunto un paio di gocce
di soluzione di fenolftaleina, che è un indicatore che cambia colore quando si supera il punto di equivalenza
in modo da avere un riscontro visivo di quello che succede nel beker.

Mediante la buretta si è poi versato nel beker la soluzione 0.1 M (valore di cui abbiamo trascurato incertezza)
di \ce{NaOH}, inizialmente a
passi di 0.5 ml, poi, in prossimità del punto equivalente, goccia per goccia. Abbiamo annotato per ogni passo, valore
del pH (misurato col pHmetro) e volume totale di soluzione di idrossido di sodio versato. L'incertezza sul pH è 0.003
(incertezza tipo di risoluzione), mentre quella sul volume versato, che è la differenza tra due letture di volume,
vale 0.02 ml. 

Dopo aver raggiunto il punto di equivalenza abbiamo continuato a versare soluzione di \ce{NaOH} per
ottenere una curva di titolazione completa. Ci siamo fermati poco prima che la soluzione raggiungesse pH = 11,
per evitare di rovinare l'elettrodo del pHmetro.
