\section*{Materiale}

\subsection*{Materiale per la costruzione delle celle galvaniche}

\begin{itemize}
        \setlength{\parskip}{0pt}
        \item{Vetreria: 2 beker da 100 ml, uno da 600 ml per il recupero delle acque di lavaggio, 
            pipetta da 10 ml con aspiratore, pipette Pasteur,
            matraccio da 50 ml, 2 bottiglie da 20 ml;}
        \item{Multimetro, elettrodi di Cu, Zn e Ag,
            sostegno per elettrodi, led per il funzionamento della pila e cavi a banana;}
        \item{Ponte salino (soluzione 2 M di \ce{KNO3} in agar);}
        \item{Spruzzetta con acqua distillata;}
        \item{Soluzioni \ce{H2SO4} o \ce{HCl} 1 M, soluzioni madre 0.5 M di \ce{CuSO4}, \ce{ZnSO4}, \ce{AgNO3};}
        \item{Carta vetrata.}
\end{itemize}

\subsection*{Materiale per la determinazione di \ce{Ca^{2+}} con elettrodo ionoselettivo}

\begin{itemize}
        \setlength{\parskip}{0pt}
        \item{Vetreria: 2 beker da 100 ml e 3 da 50ml, 2 pipette da 2 ml, 3 pipette da 10 ml, pipette Pasteur,
            propipetta, cilindro da 100 ml;}
        \item{Sonde elettrochimiche con relativo supporto, lettore di sonde Mettler-Toledo;}
        \item{Agitatore magnetico con ancorette (una per soluzione);}
        \item{Termometro;}
        \item{Acqua distillata;}
        \item{Soluzione madre di concentrazione 1 g/l di ioni \ce{Ca^{2+}}, soluzione ISA (\emph{Ion Stranght
            Adjuster});}
        \item{Campioni di misura: acqua minerali differenti, acque di acquedotti diversi, succhi di frutta, latte;}
        \item{Francesco Pasa, Andrea Miani, Davide Bazzanella e tanta, tanta allegria.}
\end{itemize}

\section*{Procedura e incertezze}

\subsection*{Misura del potenziale di alcune celle galvaniche}

In questa parte dell'esperienza abbiamo costruito 3 celle galvaniche: Cu-Zn, Ag-Cu, Ag-Zn. Abbiamo verificato
le previsioni fatte con i potenziali standard di semicella. Inoltre abbiamo provato a accendere un led con
due celle galvaniche in serie.

Abbiamo cominciato col preparare il materiale occorrente. Poichè gli elettrodi erano già stati usati abbiamo
provveuto a pulirli con la carta vetrata, per liberarli dalle incrostazioni.

Successivamente sono state preparate le soluzioni necessarie per le celle, ovvero 50 ml di soluzione 0.1 M
in acqua per ognuno dei seguenti 3 composti: \ce{CuSO4}, \ce{ZnSO4} e \ce{AgNO3}. Per prepararle abbiamo
prelevato 10 ml di soluzione madre 0.5 M dei composti e le abbiamo diluite nel matraccio da 50 ml.
L'incertezza standard sul volume prelevato è di 0.03 ml, poichè abbiamo usato la pipetta con tacche ogni 0.1 ml.
L'errore sul volume finale è di 0.3 ml, avendo considerato un incertezza di risoluzione del matraccio
di 1 ml. La deviazione sulle concentrazioni delle soluzioni madri è state poste uguali a zero, non sapendo come sono
state preparate. Con questi dati si ha che la concentrazione finale vale $100 \pm 0.7$ \si{\milli\mol\per\litre}.

Le soluzioni sono state versate nei beker da 100 ml a due a due. Abbiamo inserito in ogni soluzione l'elettrodo
del materiale corrispondente (di rame nella soluzione di \ce{CuSO4}, di zinco in quella di \ce{ZnSO4} e di
argento nella \ce{AgNO3}). Abbiamo collegato gli elettrodi al multimetro digitale per misurare la f.e.m..
Infine, per far procedere la reazione redox, che porta ad uno scambio di elettroni tra le due soluzioni,
è necessario usare un ponte salino. Infatti la reazione che avviene è la seguente:

\begin{equation}
    \begin{array}{l l}
        \ce{Cu^{2+}_{(aq)} + 2e- -> Cu_{(s)}} & \text{riduzione al catodo} \\
        \ce{Zn_{(s)} -> Zn^{2+}_{(aq)} + 2e-} & \text{ossidazione all'anodo} \\
    \end{array}
\end{equation}

Poichè le reazioni avvengono in beker separati, i due elettroni che perde lo zinco devono passare
attraverso gli elettrodi e il multimetro (o qualsiasi dispositivo io voglia alimentare)
per giungere nell'altro beker e ridurre il rame
(in realtà è un effetto a cascata, ma per semplicità si può considerare un passaggio). Questa reazione 
tende però a caricare le due soluzioni attraverso lo spostamento di elettroni. La differenza di potenziale che
si crea si oppone però alla reazione che giunge subito all'equilibrio e poi si arresta. Per evitare questo problema
è necessario usare il ponte salino menzionato sopra, che non è altro che una soluzione di \ce{KNO3} in agar (una gelatina) in un tubetto a contatto con entrambe le soluzioni. Il sale si dissocia in \ce{K+} e \ce{NO3-} nel tubetto.
Quando le soluzioni tendono a caricarsi, gli ioni del ponte migrano e si separano. Gli ioni \ce{K+} finiscono
nella soluzione con il solfato di rame, mentre gli \ce{NO3-} in quella con lo zinco solfato. Questo ristabilisce l'equilibro e permette alla redox di procedere.

La reazione illustrata vale per la coppia rame-zinco, ma per le altre coppie la reazione è simile. 

Per ogni coppia dopo aver immerso il ponte salino abbiamo misurato la f.e.m., verificando che fosse
uguale a quella prevista dai potenziali standard di semicella. Per concludere, abbiamo collegato due e poi anche tre
celle in serie (con l'aiuto degli altri gruppi) e abbiamo provato a chiudere il circuito con un LED, verificando
che il diodo emetteva un po' di luce. 
\subsection*{Taratura degli elettrodi ionoselettivi (ISE)}
Per poter tarare il nostro strumento abbiamo preparato delle soluzioni a concentrazione nota di \ce{Ca2+} attraverso 3 diluizioni successive in proporzioni 1:10 di una soluzione madre di concentrazione 1000mg/l. 
Per prepararle abbiamo prelevato 10 ml di soluzione madre e l'abbiamo diluita nel matraccio da 100 ml.%Giusto?
L'incertezza standard sul volume prelevato è di 0.03 ml, poichè abbiamo usato la pipetta con tacche ogni 0.1 ml.
L'errore sul volume finale della prima diluizione è di 0.3 ml, avendo considerato un incertezza di risoluzione del matraccio
di 1 ml. La deviazione sulle concentrazioni della soluzione madre è stata posta uguale a zero, non sapendo come è
stata preparata.
Operando analogamente, partendo da questa soluzione figlia abbiamo ottenuto le altre 2 soluzoni necessarie per la taratura. 
Una volta ottenute le 4 soluzioni desiderate s'è impostata la temperatura d'esercizio, per permettere al lettore di fornirci misure dell'attività di \ce{Ca2+} corrette rispetto alle possibili variazioni dovute alla temperatura.