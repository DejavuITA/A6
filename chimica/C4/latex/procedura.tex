\section*{Procedura e incertezze}

\subsection*{Misura del potenziale di alcune celle galvaniche}

Nella prima parte dell'esperienza abbiamo costruito tre celle galvaniche (Cu-Zn, Ag-Cu, Ag-Zn) verificando le previsioni fatte con i potenziali standard di semicella.
Inoltre abbiamo provato ad accendere un led con una o più celle galvaniche in serie.
\\

Poichè gli elettrodi erano già stati usati, la prima operazione è stata quella di pulirli con la carta vetrata, per liberarli dalle incrostazioni.

Successivamente sono state preparate le soluzioni necessarie per le celle, ovvero 50 ml di soluzione 0.1 M in acqua per ognuno dei seguenti tre composti: \ce{CuSO4}, \ce{ZnSO4} e \ce{AgNO3}.
Per prepararle abbiamo prelevato 10 ml di soluzione madre 0.5 M di ciascun composto e le abbiamo diluite nel matraccio da 50 ml, ottenendo la concentrazione voluta.
L'incertezza standard sul volume prelevato è di 0.03 ml, poichè abbiamo usato la pipetta con tacche ogni 0.1 ml.
L'errore sul volume finale è pertanto di 0.3 ml, avendo considerato un incertezza di risoluzione del matraccio di 1 ml.
La deviazione sulle concentrazioni delle soluzioni madri è stata poste uguali a zero, non sapendo come sono
state preparate.
Con questi dati si ha che la concentrazione finale vale $100 \pm 0.7$ \si{\milli\mol\per\litre}.

Le soluzioni sono state versate in beker da 100 ml distinti a due a due, creando tre possibili combinazioni.
Abbiamo inserito in ogni soluzione l'elettrodo del materiale corrispondente (di rame nella soluzione di \ce{CuSO4}, di zinco in quella di \ce{ZnSO4} e di argento nella \ce{AgNO3}).
Abbiamo collegato gli elettrodi al multimetro digitale per misurare la forza elettromotrice.
Nella situazione così creata, però, il volt-metro mostrava una differenza di potenziale ai capi degli elettrodi nulla.
Osserviamo per esempio le reazioni nel caso dei due elettrodi di rame e zinco:

\begin{equation}
    \begin{array}{l l}
        \ce{Cu^{2+}_{(aq)} + 2e- -> Cu_{(s)}} & \text{riduzione al catodo} \\
        \ce{Zn_{(s)} -> Zn^{2+}_{(aq)} + 2e-} & \text{ossidazione all'anodo} \\
    \end{array}
\end{equation}

Poichè le reazioni avvengono in beker separati, gli elettroni che perde lo zinco rimangono in uno dei due beker, mentre gli elettroni acquisiti dal rame aumentano la carica dell'altro beker.
Ciò provoca uno scompenso di carica con una propria differenza di potenziale, la cui azione si oppone alla forza elettromotrice generata dalla reazione di ossidoriduzione. Le due azioni, raggiungendo l'equilibrio, si arrestano.
Per per far procedere la reazione redox è necessario usare il ponte salino, nel nostro caso è costituito da una soluzione di \ce{KNO3} in agar (una gelatina) in un tubetto a contatto con entrambe le soluzioni.
Il sale contenuto nel ponte salino si dissocia in \ce{K+} e \ce{NO3-} nel tubetto.
Quando le soluzioni tendono a caricarsi, gli ioni del ponte migrano e si separano.
Gli ioni \ce{K+} finiscono nella soluzione con il solfato di rame, mentre gli \ce{NO3-} in quella con lo zinco solfato.
Questo ristabilisce l'equilibro di carica per entrambe le soluzioni nei beker e permette alla reazioni di ossidoriduzione di procedere.

L'esempio era riferito alla coppia rame-zinco, ma il ragionamento per le altre coppie è lo stesso.

Per ogni coppia dopo aver immerso il ponte salino abbiamo misurato la forza elettromotrice, verificando che fosse uguale a quella prevista dai potenziali standard di semicella.
\\

Per concludere, abbiamo collegato (con l'aiuto di altri gruppi) due e poi anche tre celle in serieper alimentare un LED, verificando che il diodo emetteva luce.

\subsection*{Taratura degli elettrodi ionoselettivi (ISE)}

Per poter tarare il nostro strumento abbiamo preparato delle soluzioni a concentrazione nota di \ce{Ca2+} attraverso 3 diluizioni successive in proporzioni 1:10 di una soluzione madre di concentrazione 1000mg/l. 
Per prepararle abbiamo prelevato 10 ml di soluzione madre e l'abbiamo diluita nel matraccio da 100 ml.%Giusto?
L'incertezza standard sul volume prelevato è di 0.03 ml, poichè abbiamo usato la pipetta con tacche ogni 0.1 ml.
L'errore sul volume finale della prima diluizione è di 0.3 ml, avendo considerato un incertezza di risoluzione del matraccio di 1 ml.
La deviazione sulle concentrazioni della soluzione madre è stata posta uguale a zero, non sapendo come è stata preparata.
Operando analogamente, partendo da questa soluzione figlia abbiamo ottenuto le altre 2 soluzoni necessarie per la taratura. 
Una volta ottenute le 4 soluzioni desiderate s'è impostata la temperatura d'esercizio, per permettere allo strumento di fornirci misure dell'attività di \ce{Ca2+} corrette rispetto alle possibili variazioni dovute alla temperatura.