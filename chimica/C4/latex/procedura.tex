\section*{Procedura e incertezze}

\subsection*{Misura del potenziale di alcune celle galvaniche}

In questa parte dell'esperienza abbiamo costruito 3 celle galvaniche: Cu-Zn, Ag-Cu, Ag-Zn. Abbiamo verificato
le previsioni fatte con i potenziali standard di semicella. Inoltre abbiamo provato a accendere un led con
due celle galvaniche in serie.

Abbiamo cominciato col preparare il materiale occorrente. Poichè gli elettrodi erano già stati usati abbiamo
provveuto a pulirli con la carta vetrata, per liberarli dalle incrostazioni.

Successivamente sono state preparate le soluzioni necessarie per le celle, ovvero 50 ml di soluzione 0.1 M
in acqua per ognuno dei seguenti 3 composti: \ce{CuSO4}, \ce{ZnSO4} e \ce{AgNO3}. Per prepararle abbiamo
prelevato 10 ml di soluzione madre 0.5 M dei composti e le abbiamo diluite nel matraccio da 50 ml.
L'incertezza standard sul volume prelevato è di 0.03 ml, poichè abbiamo usato la pipetta con tacche ogni 0.1 ml.
L'errore sul volume finale è di 0.3 ml, avendo considerato un incertezza di risoluzione del matraccio
di 1 ml. La deviazione sulle concentrazioni delle soluzioni madri è state poste uguali a zero, non sapendo come sono
state preparate. Con questi dati si ha che la concentrazione finale vale $100 \pm 0.7$ \si{\milli\mol\per\litre}.

Le soluzioni sono state versate nei beker da 100 ml a due a due. Abbiamo inserito in ogni soluzione l'elettrodo
del materiale corrispondente (di rame nella soluzione di \ce{CuSO4}, di zinco in quella di \ce{ZnSO4} e di
argento nella \ce{AgNO3}). Abbiamo collegato gli elettrodi al multimetro digitale per misurare la f.e.m..
Infine, per far procedere la reazione redox, che porta ad uno scambio di elettroni tra le due soluzioni,
è necessario usare un ponte salino. Infatti la reazione che avviene è la seguente:

\begin{equation}
    \begin{array}{l l}
        \ce{Cu^{2+}_{(aq)} + 2e- -> Cu_{(s)}} & \text{riduzione al catodo} \\
        \ce{Zn_{(s)} -> Zn^{2+}_{(aq)} + 2e-} & \text{ossidazione all'anodo} \\
    \end{array}
\end{equation}

Poichè le reazioni avvengono in beker separati, i due elettroni che perde lo zinco devono passare
attraverso gli elettrodi e il multimetro (o qualsiasi dispositivo io voglia alimentare)
per giungere nell'altro beker e ridurre il rame
(in realtà è un effetto a cascata, ma per semplicità si può considerare un passaggio). Questa reazione 
tende però a caricare le due soluzioni attraverso lo spostamento di elettroni. La differenza di potenziale che
si crea si oppone però alla reazione che giunge subito all'equilibrio e poi si arresta. Per evitare questo problema
è necessario usare il ponte salino menzionato sopra, che non è altro che una soluzione di \ce{KNO3} in agar (una gelatina) in un tubetto a contatto con entrambe le soluzioni. Il sale si dissocia in \ce{K+} e \ce{NO3-} nel tubetto.
Quando le soluzioni tendono a caricarsi, gli ioni del ponte migrano e si separano. Gli ioni \ce{K+} finiscono
nella soluzione con il solfato di rame, mentre gli \ce{NO3-} in quella con lo zinco solfato. Questo ristabilisce l'equilibro e permette alla redox di procedere.

La reazione illustrata vale per la coppia rame-zinco, ma per le altre coppie la reazione è simile. 

Per ogni coppia dopo aver immerso il ponte salino abbiamo misurato la f.e.m., verificando che fosse
uguale a quella prevista dai potenziali standard di semicella. Per concludere, abbiamo collegato due e poi anche tre
celle in serie (con l'aiuto degli altri gruppi) e abbiamo provato a chiudere il circuito con un LED, verificando
che il diodo emetteva un po' di luce. 

\subsection*{Taratura dell'elettrodo ionoselettivo (ISE)}

Per poter tarare il nostro strumento abbiamo preparato tre soluzioni di concentrazione 1, 10 e 100 mg/l
di \ce{Ca^{2+}} partendo da una soluzione madre di concentrazione 1000 mg/l. 
Al fine di evitare troppe propagazioni gli errori, abbiamo preparato le due soluzioni 10 mg/l e 100 mg/l
attraverso diluizioni diretta dalla soluzione madre e non attraverso diluizioni successive.
Abbiamo prelevato rispettivamente 1 ml e 10 ml di soluzione madre e l'abbiamo diluita nel matraccio da 100 ml.
L'incertezza standard sul volume prelevato è di 0.03 ml per la soluzione da 100 mg/l e di 0.006 per quella da 10 mg/l,
poichè abbiamo usato pipette con tacche ogni 0.1 ml e 0.02 ml rispettivamente.
L'errore sul volume finale della prima diluizione è di 0.3 ml,
avendo considerato un incertezza di risoluzione del matraccio di 1 ml.
La deviazione sulle concentrazioni della soluzione madre è stata posta uguale a zero, non sapendo come è
stata preparata. Quindi i valori finali di concentrazione per queste due soluzioni di taratura sono
$100 \pm 0.4$ mg/l e $10 \pm 0.07$ mg/l.

Operando analogamente, partendo dalla soluzione figlia di concentrazione 100 mg/l abbiamo ottenuto l'altra soluzone
necessaria per la taratura, prendendo 1 ml e diluendolo nel matraccio da 100 ml. Con incertezze uguali alle precedenti (unica differenza il fatto che la concentrazione di partenza ha una sua incertezza), si ottiene il seguente
valore di concentrazione: 1 $\pm$ 0.009 mg/l.

Una volta ottenute le 3 soluzioni desiderate s'è impostata la temperatura d'esercizio nelle impostazioni
dello strumento, per permettere al lettore di fornirci misure dell'attività di \ce{Ca^{2+}}
corrette rispetto alle possibili variazioni dovute alla temperatura. Abbiamo dunque eseguito le istruzioni a video
per la taratura dello strumento.

\subsection*{Misura della concentrazione di \ce{Ca^{2+}} di campioni incogniti}

Dopo aver tarato l'elettrodo, abbiamo misurato la concentrazione di \ce{Ca^{2+}} di vari campioni di acqua,
sia in bottiglia (Panna, Guizza e Ferrarelle) che di acquedotto (Piné), succo di frutta e latte.

Per la misura degli ioni calcio nelle varie acque, abbiamo semplicemente versato l'acqua nel becker
e aggiunto 2\% in volume di soluzione ISA, che serve per portare la forza ionica della soluzione
a livelli alti. In questo modo la quantità di calcio in soluzione non varia di molto la forza ionica,
e quindi la misura è più precisa poichè gli elettrodi misurano l'attività e non la concentrazione.

Poi abbiamo inserito gli elettrodi nell'acqua, mescolato per un minuto circa, spento l'agitatore magnetico,
atteso un po' che la soluzione entrasse in equilibrio e poi preso la misura. In modo analogo abbiamo
misurato il calcio presente nel latte e nel succo. Il latte è stato prima diluito di un fattore 100,
mentre il succo è stato diluito di un fattore 4, entrambi con procedimenti analoghi a quelli spiegati nel paragrafo
precedente e col matraccio da 100 ml.
