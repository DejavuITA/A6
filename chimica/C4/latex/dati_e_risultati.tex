\section*{Dati e risultati}

\subsection*{Celle galvaniche}

È possibile calcolare a priori la f.e.m. di una certa cella galvanica utilizzando i potenziali di Nerst.
Ogni metallo ha una diversa tendenza ad ossidarsi e un corrispondente potenziale di Nerst $E^\circ$, che indica la
forza elettromotrice che il metallo genererebbe se collegato ad un elettrodo di idrogeno (elemento scelto come
standard) a 25$^\circ$C. Per ottenere la f.e.m. di una generica cella galvanica basta sottrarre i potenziali di
Nerst dei metalli usati; inoltre quello con il potenzione più alto è l'anodo (+), mentre quello più basso
il catodo (-). Riportiamo in Tabella \ref{tab:nerst} i potenziali dei metalli usati:

\begin{table}[h]
    \centering
    \begin{tabular}{l c}
    \toprule
    Metallo & $E^\circ$ [V] \\
    \midrule
    \ce{Ag} & 0.8 \\
    \ce{Cu} & 0.34 \\
    \ce{Zn} & -0.76 \\
    \bottomrule
    \end{tabular}
    \caption{Potenziali di Nerst per i metalli usati.}
    \label{tab:nerst}
\end{table}

Immergendo l'elettrodo di \ce{Cu} nella soluzione di \ce{CuSO4} e collegandolo, mediante un voltmetro, all'elettrodo
di \ce{Zn} immerso nella soluzione di \ce{ZnSO4}, abbiamo verificato che, una volta inserito il ponte salino, tra i due materiali si trovava una forza elettromotrice di 1.08 V $\pm$ 0.02 V (l'incertezza è stata assegnata notado di quanto variava il valore sul monitor del multimetro). Osservando il segno riportato dal multimetro abbiamo verificato che l'elettrodo di \ce{Cu} era l'anodo (+), mentre quello \ce{Zn} era il catodo (-).
Sia il valore ottenuto che il segno degli elettrodi è in accordo con i potenziali di Nerst, che prevedono
1.10 V per la coppia di materiali usati.

Sostituendo allo zinco l'elettrodo di argento, immerso nella soluzione di \ce{AgNO3} abbiamo misurato
una differenza di potenziale di 0.43 $\pm$ 0.01 V. In questo caso l'\ce{Ag} fungeva da anodo (+), mentre
il rame era il catodo (-).

Di nuovo, usando nelle rispettive soluzioni lo zinco e l'argento e procedendo in modo analogo,
abbiamo misurato la più alta differenza di potenziale pari a 1.50 $\pm$ 0.01 V,
con l'\ce{Ag} all'anodo (+) e lo \ce{Zn} come catodo (-).

Mettendo in serie rispettivamente 2 e 3 celle galvaniche di questo tipo, con i gruppi vicini,
abbiamo ottenuto: 2.94 $\pm$ 0.01 V e 4.40 $\pm$ 0.01 V. Con queste tensioni abbiamo provato ad alimentare un
diodo LED, che siamo riusciti a far funzionare, anche se emetteva pochissima luce. Evidentemente
le celle non sono in grado di fornire tensione o corrente a sufficienza. Il LED necessita di circa 100 mA
che si ottengono applicando 10 V ai suoi capi.

Per facilità di lettura riportiamo i dati nella Tabella \ref{tab:celle}. Siamo felici di notare che i dati concordano bene
con i potenziali standard, nonostante la temperatura fosse di 27$^\circ$C invece che 25$^\circ$C. 

\begin{SCtable}[1][h]
    \centering
    \begin{tabular}{l c c}
        \toprule
        \multicolumn{1}{c}{Cella} & \multicolumn{2}{c}{f.e.m. [V]} \\
        (anodo-catodo) & Nerst & Misurate \\
        \midrule
        Cu-Zn & 1.10 & 1.08 \\
        Ag-Cu & 0.46 & 0.43 \\
        Ag-Zn & 1.56 & 1.50 \\
        Ag-Zn (2x) & 3.12 & 2.94 \\
        Ag-Zn (3x) & 4.68 & 4.40 \\
        \bottomrule
    \end{tabular}
    \caption{Forze elettromotrici misurate con il multimetro. Nella prima colonna abbiamo riportato le coppie di metalli
    secondo la convezione anodo-catodo. I (2x) e (3x) indicano il numero di celle in serie. Le incertezze sono di 0.01 V.}
    \label{tab:celle}
\end{SCtable}

\subsection*{Misura della concentrazione di \ce{Ca^{2+}}}

Abbiamo misurato diverse acque, del latte e un succo di frutta, confrontando, ove possibile i valori ottenuti con i dati riportati sull'etichetta. Nella seguente tabella sono esposti i risultati:

\begin{table}
    \centering
    \footnotesize

    \begin{tabular}{r @{\quad} c c c c c c c c}
    \multicolumn{9}{c}{\textbf{Concentrazioni di \ce{Ca^{2+}} in acqua, latte e succo}} \\
        \toprule
        & \multicolumn{4}{c}{Acqua} & Latte & Latte & Succo & Succo \\
        & Guizza & Panna & Ferrarelle & di Piné & diluito 1:100 & non diluito & diluito 1:4 & non diluito \\
         \midrule
         \phantom{.}[\ce{Ca^{2+}}] (mg/l) & 14.3 & 33.4 & 460 & 14.8 & 10.5 & 1050 $\pm$ 8 & 10.6 & 42 $\pm$ 2 \\
         ddp (\si{\milli\volt}) & 15.1 & 25.2 & 56 & 15.5 & 11.5 & - & 11.6 & - \\
         Durezza [$^\circ f$] & 3.6 $\pm$ 0.2 & 8.4 $\pm$ 0.3 & 115 $\pm$ 5 & 3.7 $\pm$ 0.2 & 2.6 $\pm$ 0.1 & - & 2.7 $\pm$ 0.1 & - \\
         \midrule
         \phantom{.}[\ce{Ca^{2+}}] etichetta & \multirow{2}{*}{10.6} & \multirow{2}{*}{32}
         & \multirow{2}{*}{392} & \multirow{2}{*}{-} & \multirow{2}{*}{-} & \multirow{2}{*}{1200} & \multirow{2}{*}{-} & \multirow{2}{*}{35} \\
        (mg/l) &   &  &  &  &  &  \\
        \bottomrule
    \end{tabular}
    \caption{La tabella riporta i dati misurati di concentrazione di \ce{Ca^{2+}} per vari campioni. Sono riportati sia
        i valori di concentrazione sia quelli di tensione misurata dal lettore. Inoltre per le acque in bottiglia,
        sono riportati i dati forniti dall'etichetta, mentre per latte e succo abbiamo indicato un valore tipico. Le incertezze
    dove non esplicitate sono uguali a 0.1 mg/l per le concentrazione e a 0.1 mV per le tensioni.}
    \label{tab:vaccino}
\end{table}

Abbiamo misurato erroneamente, le concentrazioni di \ce{Ca2+} in soluzioni prive di ISA ottenendo risultati molto distanti da quelli nominali. Successivamente, una volta scoperto l'errore commesso, abbiamo preso delle nuove misure, le quali risultano molto vicine al valore nominale in nostro possesso.
Per l'acqua Guizza la quale riporta in etichetta una concentrazione di \ce{Ca2+} uguale a $10.6 \text{mg/l}$ mentre senza ISE otteniamo un valore di $26.8 \text{mg/l}$ con l'ISE si ottiene $14.3 \text{mg/l}$.
Analogamente con il latte, il quale è stato diluito 100 volte, abbiamo misurato senza ISA $6.34 \text{mg/l}$ mentre con si ha 10.5 mg/l. Il valore nominale, non reperibile sull'etichetta è stato ricercato nel web. Abbiamo trovato che un valore tipico è $1200 \text{mg/l}$ la quale, appunto, è molto vicina a quella misurata.
