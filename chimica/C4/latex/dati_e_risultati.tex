\section*{Dati e risultati}

\subsection*{Celle galvaniche}

È possibile calcolare a priori la f.e.m. di una certa cella galvanica utilizzando i potenziali di Nerst.
Ogni metallo ha una diversa tendenza ad ossidarsi e un corrispondente potenziale di Nerst $E^\circ$, che indica la
forza elettromotrice che il metallo genererebbe se collegato ad un elettrodo di idrogeno (elemento scelto come
standard) a 25$^\circ$C. Per ottenere la f.e.m. di una generica cella galvanica basta sottrarre i potenziali di
Nerst dei metalli usati; inoltre quello con il potenzione più alto è l'anodo (+), mentre quello più basso
il catodo (-). Riportiamo in Tabella \ref{tab:nerst} i potenziali dei metalli usati:

\begin{table}[h]
    \centering
    \begin{tabular}{l c}
    \toprule
    Metallo & $E^\circ$ [V] \\
    \midrule
    \ce{Ag} & 0.8 \\
    \ce{Cu} & 0.34 \\
    \ce{Zn} & -0.76 \\
    \bottomrule
    \end{tabular}
    \caption{Potenziali di Nerst per i metalli usati.}
    \label{tab:nerst}
\end{table}

Immergendo l'elettrodo di \ce{Cu} nella soluzione di \ce{CuSO4} e collegandolo, mediante un voltmetro, all'elettrodo
di \ce{Zn} immerso nella soluzione di \ce{ZnSO4}, abbiamo verificato che, una volta inserito il ponte salino, tra i due materiali si trovava una forza elettromotrice di 1.08 V $\pm$ 0.02 V (l'incertezza è stata assegnata notado di quanto variava il valore sul monitor del multimetro). Osservando il segno riportato dal multimetro abbiamo verificato che l'elettrodo di \ce{Cu} era l'anodo (+), mentre quello \ce{Zn} era il catodo (-).
Sia il valore ottenuto che il segno degli elettrodi è in accordo con i potenziali di Nerst, che prevedono
1.10 V per la coppia di materiali usati.

Sostituendo allo zinco l'elettrodo di argento, immerso nella soluzione di \ce{AgNO3} abbiamo misurato
una differenza di potenziale di 0.43 $\pm$ 0.01 V. In questo caso l'\ce{Ag} fungeva da anodo (+), mentre
il rame era il catodo (-).

Di nuovo, usando nelle rispettive soluzioni lo zinco e l'argento e procedendo in modo analogo,
abbiamo misurato la più alta differenza di potenziale pari a 1.50 $\pm$ 0.01 V,
con l'\ce{Ag} all'anodo (+) e lo \ce{Zn} come catodo (-).

Mettendo in serie rispettivamente 2 e 3 celle galvaniche di questo tipo, con i gruppi vicini,
abbiamo ottenuto: 2.94 $\pm$ 0.01 V e 4.40 $\pm$ 0.01 V. Con queste tensioni abbiamo provato ad alimentare un
diodo LED, che siamo riusciti a far funzionare, anche se emetteva pochissima luce. Evidentemente
le celle non sono in grado di fornire tensione o corrente a sufficienza. Il LED necessita di circa 100 mA
che si ottengono applicando 10 V ai suoi capi.

\subsection*{Misura della concentrazione di \ce{Ca^{2+}}}
