\section*{Materiale}

\begin{itemize}
        \setlength{\parskip}{-1pt}
        \item{Vetreria: buretta, beker, matracci da 10 ml e 25 ml,
            contenitori per le varie soluzioni, cuvette per il fotospettrometro,
        \item{pHmetro;}
            pipette Pasteur;}
        \item{\SI{100}{\milli\liter} di \ce{CH3COOH} a concentrazione incognita;}
        \item{fenolftaleina e \ce{NaOH} a badilate}
        \item{Francesco Pasa, Andrea Miani, Davide Bazzanella e una Pepsi™.}
\end{itemize}

\section*{Procedura e incertezze}

Per prima cosa abbiamo calibrato il pHmetro con tre soluzioni tampone di pH noto e costante rispettivamente di 4.01, 7.00 e 9.??.
Una volta 

%\begin{equation*}
%	\ce{CH3COOH\ped{(aq)} + H2O\ped{(aq)}} \rightleftarrows \ce{CH3COO-\ped{(aq)} + H3O+\ped{(aq)}}
%\end{equation*}