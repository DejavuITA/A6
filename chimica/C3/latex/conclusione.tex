\section*{Conclusione}

Possiamo dire che l'esperienza è stata portata a termine positivamente.
Abbiamo utilizzato il pHmetro per analizzare il pH della soluzione a concentrazione incognita di \ce{CH3COOH} in funzione dell'aggiunta di una soluzione basica 0.1 M di \ce{NaOH} e in tal modo determinare la titolazione di quella incognita.
Il valore ottenuto per la concentrazione della soluzione incognita è di $10.15 \pm 0.04$ \si{\milli\mol\per\litre} ed è in linea con le aspettative in quanto è contenuto nel range tale per cui $\gamma_{a} \simeq 1$.

\section*{Bonus: pH della Pepsi™}

Con una lattina di Pepsi™ comprata ai distributori automatici fuori dal laboratorio, abbiamo prodotto e misurato con il pHmetro prima una soluzione diluita al 10\% in volume e poi una soluzione non diluita di Pepsi™ .
I valori ottenuti sono stati:
$$pH_{10\%} = 3.4$$
$$pH_{100\%} = 2.2$$
