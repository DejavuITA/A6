\section*{Scopo}

Come nell'esperienza precedente, anche in questo caso lo scopo è titolare una
soluzione di un acido (nel nostro caso \ce{CH3COOH}) la cui concentrazione è incognita.
In questo caso, la titolazione è stata eseguita mediante misure di pH, ovvero misurando la concentrazione di ioni \ce{H3O+}.
Aggiungendo una soluzione basica è possibile, estrapolando l'andamento del pH in funzione del volume di soluzione basica aggiunta di cui si conosce la concentrazione, ricavarne la titolazione della soluzione incognita.
