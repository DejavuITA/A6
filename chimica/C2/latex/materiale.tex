\section*{Materiale}

\begin{itemize}
        \setlength{\parskip}{-1pt}
        \item{Vetreria: buretta, beker, matracci da 10 ml e 25 ml,
            contenitori per le varie soluzioni, cuvette per il fotospettrometro,
            pipette Pasteur;}
        \item{Solfato di rame (\ce{CuSO4}), acqua distillata e soluzione incognita;}
        \item{Fotospettrometro;}
        \item{Francesco Pasa, Andrea Miani, Davide Bazzanella, uno dei quali camice-munito.}
\end{itemize}

\section*{Procedura e incertezze}

Per prima cosa abbiamo preparato, dalla soluzione madre $0.5 \pm 0.001$ M a
nostra disposizione, $25 \pm 0.3$ ml di soluzione figlia 0.16 M
di \ce{CuSO4}. Avendo prelevato $8.0 \pm 0.03$ ml di soluzione madre, risulta che
la concentrazione della soluzione figlia vale $0.160 \pm 0.002 $ M. Abbiamo sempre
usato incertezze standard.

Per ricavare la concentrazione della soluzione incognita, si usa la legge di legge
di Lambert-Beer, che indica come varia l'assorbanza di una soluzione in funzione
della concentrazione e della lunghezza della sezione di soluzione attraversata dalla luce.
L'assorbanza $A$ è un numero puro ed è definita come:

\begin{equation}
    A = - \log_{10}{\frac{I}{I_0}}
    \label{eq:ass}
\end{equation}
%
dove $I_0$ è l'intensità della luce trasmessa dall'acqua, mentre $I$ quella trasmessa
dalla soluzione in esame.

In termini di assorbanza la legge di Lamber-Beer assume la forma:

\begin{equation}
    A = \epsilon \ell c = \alpha c
    \label{eq:lambeer}
\end{equation}
%
dove $\epsilon$ è l'assorbanza molare, $\ell$ la lunghezza della sezione di soluzione
attraversata dalla luce e $c$ la concentrazione molare. Nel nostro caso $\ell$ è costante
ed è data dalla lunghezza della cuvetta, per cui eseguiremo i calcoli in termini di
$\alpha$. In nostro scopo è quindi ricavare $\alpha$ con delle soluzioni a concentrazione
nota, in modo da poter poi ricavare $c$ conoscendo l'assorbanza della soluzione
incognita.

La legge è dunque una retta che passa per l'origine. Questa legge vale per basse
concentrazioni, come regola empirica si considera che $A < 1$, per cui è necessario
usare soluzioni poco concentrate affinche la titolazione risulti corretta.

A questo punto abbiamo misurato la curva di assorbimento della soluzione con il
fotospettrometro, confrontandolo con quello dell'acqua distillata. Questo passo
serve per ottenere la lunghezza d'onda in cui si trova il massimo di assorbimento,
che è risultato essere a $\lambda\ped{max} =$ 825 nm\footnote{In realtà la lunghezza
d'onda del massimo di assorbimento varia di qualche nanometro con la concentrazione,
tuttavia noi abbiamo mantenuto fisso $\lambda\ped{max} =$ 825 nm}, ed è inoltre necessario per
assicurarsi che la concentrazione resti nell'intervallo di validità della Lambert-Beer.

Poichè la soluzione figlia 0.16 M aveva un'assorbanza $A \simeq 2$ a $\lambda\ped{max}$,
oltre l'intervallo di validità della legge, abbiamo deciso di usare soluzioni
0.016 M, 0.032 M, 0.048 M, 0.064 M per ricavare $\alpha$. In questo modo, si dovrebbero
ottenere curve di assorbimento con assorbanze di picco di circa 0.2, 0.4, 0.6, 0.8,
equispaziate e contenute nell'intervallo di validità.

Abbiamo quindi eseguito le diluizioni necessarie, avendo cura di lavare la buretta con
un po' di soluzione per evitare diluizioni non volute. Sono state preparate 4 soluzioni
da 10 ml, le cui concentrazioni valgono $0.016 \pm 0.001$ M, $0.032 \pm 0.001$ M,
$0.048 \pm 0.002$ M e $0.064 \pm 0.002$ M.

Abbiamo quindi misurato le curve di assorbimento delle quattro soluzioni e della
soluzione incognita con il fotospettrometro. L'assorbanza è misurata relativamente a quella
dell'acqua distillata.
