\documentclass[11pt, twoside, a4paper]{article}
\usepackage[italian]{babel}
\usepackage[utf8]{inputenc}
\usepackage{amsmath}
\usepackage[cm]{fullpage}
\usepackage{graphicx}
\usepackage{booktabs}
\usepackage{wrapfig}
\usepackage{multirow}
\usepackage{sidecap}
\usepackage{siunitx}
\usepackage{array}
\usepackage[font=small]{caption}
\usepackage[bookmarks, hidelinks]{hyperref}
\usepackage{float}
\usepackage{wrapfig}
\usepackage{titlesec}
\usepackage{subcaption}
\usepackage[version=3]{mhchem}

\titlespacing\section{0pt}{10pt plus 4pt minus 4pt}{8pt plus 2pt minus 5pt}
\titlespacing\subsection{0pt}{10pt plus 4pt minus 4pt}{8pt plus 2pt minus 5pt}
\titlespacing\subsubsection{0pt}{10pt plus 4pt minus 4pt}{8pt plus 2pt minus 5pt}

\begin{document}

\begin{center}

        {\huge Soluzioni elettrolitiche: titolazioni conduttimetriche}
    \vspace{0.1cm}

      	{Francesco Pasa, Andrea Miani, Davide Bazzanella - Gruppo 8 Mercoledì} \\
      	{francescopasa@gmail.com - 26 marzo 2014}
    \vspace{-0.2cm}

\end{center}

\section*{Scopo}

L'obbiettivo di questa esperienza di laboratorio è quello di determinare la concentrazione
di una soluzione mediante misure di conduttimetria. Per fare ciò si sfutta una reazione di
precipitazione, mediante la quale è possibile sostituire gli elettroliti della soluzione
a concentrazione incognita con altri ioni (nel nostro caso abbiamo sostituito \ce{Ag^+} con \ce{Na^+}),
che hanno una conducibilità diversa. In questo modo, misurando la conducibilità, si può
estrapolare la quantità di elettroliti originari e quindi la concentrazione.

\section*{Materiale}

\begin{itemize}
        \setlength{\parskip}{-1pt}
        \item{Due becker, una buretta con relativo supporto, bacchetta di vetro, ``spatola'';}
        \item{Agitatore magnetico con ancoretta;}
        \item{Conduttimetro;}
        \item{Soluzioni di \ce{KCl} di concentrazione 0.1 M e 0.01 M per la taratura del conduttimetro,
                soluzione di \ce{AgNO3} di concentrazione da determinare, \ce{NaCl} e acqua;}
        \item{Agitatore magnetico con ancoretta;}
        \item{Francesco Pasa, Andrea Miani, Davide Bazzanella, possibilmente con un minimo di comprensione
                del da farsi.}
\end{itemize}

\section*{Procedura e incertezze}

Come prima cosa, abbiamo tarato il conduttimetro seguendo la procedura descritta nel manuale e utilizzando
le soluzioni 0.1 M e 0.01 M di \ce{KCl} a nostra dispozisione. Abbiamo quindi pulito la cella conduttimetrica per evitare contaminazioni.

In seguito, abbiamo preparato 50 mL di soluzione titolante 1.5 M di \ce{NaCl}. Abbiamo considerato un
incertezza standard\footnote{L'incertezza standard è uguale all'incertezza di risoluzione divisa per $\sqrt{3}$.} sul volume di 0.3 mL 
(si è ritenuto che il matraccio usato avesse un incertezza di risoluzione di 0.5 mL)
e sul peso di 0.003 g (era possibile leggere i centesimi di grammo mentre i millesimi fluttuavano, in questo caso abbiamo posto l'errore di risoluzione a 0.005 g).
Usando queste incertezze tipo, si ottiene che la soluzione titolante aveva una concentrazione $1.50 \pm 0.03$ M.

Abbiamo lavato la buretta con la soluzione per evitare diluizioni non volute. La buretta è stata riempita con
la soluzione rimanente ed è stata posta sopra il becker contenente 100 mL di soluzione titolanda a concentrazione incognita.
Misurando ogni volta la concentrazione con il conduttimetro, la titolante è stata versata nella soluzione di \ce{AgNO3}
a passi di 0.5 mL. In questo modo avviene la seguente reazione di precipitazione:

\begin{equation*}
        \ce{AgNO3\ped{(aq)} + NaCl\ped{(aq)} -> AgCl\ped{(s)} v + NaNO3\ped{(aq)}}
\end{equation*}

La parte interessante della reazione è che l'argento cloruro (\ce{AgCl}) precipita. Di conseguenza gli ioni \ce{Ag^+}
vengono rimossi dalla soluzione e non contribuiscono più alla sua conducibilità elettrica.
Poiché gli ioni \ce{Na^+} conducono meno dei \ce{Ag^+} la conducibilità diminuisce man mano che viene aggiunto NaCl,
finchè l'argento non è completamente precipitato. A questo punto, continuando ad aggiungere NaCl, si aggiungendo nuovi ioni \ce{Na^+} che fanno salire la conducibilità,
per cui il trend si inverte.

Conoscendo il volume della soluzione titolante necessaria per arrivare al punto di inversione, e quindi il numero di moli
di NaCl, si può risalire alla quantità di \ce{AgNO3} in soluzione e calcolare la concentrazione.

L'incertezza standard sul volume della soluzione titolanda (nel becker) è stata posta a 0.3 mL, mentre con la
buretta l'incertezza standard è di 0.03 mL. Per ottenere l'incertezza sul volume versato, che è la differenza tra due volumi,
occore comporre le incertezze. L'incertezza sul volume versato è quindi di 0.04 mL.

\section*{Dati e risultati}

\subsection*{Celle galvaniche}

È possibile calcolare a priori la f.e.m. di una certa cella galvanica utilizzando i potenziali di Nerst.
Ogni metallo ha una diversa tendenza ad ossidarsi e un corrispondente potenziale di Nerst $E^\circ$, che indica la
forza elettromotrice che il metallo genererebbe se collegato ad un elettrodo di idrogeno (elemento scelto come
standard) a 25$^\circ$C. Per ottenere la f.e.m. di una generica cella galvanica basta sottrarre i potenziali di
Nerst dei metalli usati; inoltre quello con il potenziale più alto è l'anodo (+), mentre quello più basso
il catodo (-). Riportiamo in Tabella \ref{tab:nerst} i potenziali dei metalli usati:

\begin{table}[h]
    \centering
    \begin{tabular}{l c}
    \toprule
    Metallo & $E^\circ$ [V] \\
    \midrule
    \ce{Ag} & 0.8 \\
    \ce{Cu} & 0.34 \\
    \ce{Zn} & -0.76 \\
    \bottomrule
    \end{tabular}
    \caption{Potenziali di Nerst per i metalli usati.}
    \label{tab:nerst}
\end{table}

Immergendo l'elettrodo di \ce{Cu} nella soluzione di \ce{CuSO4} e collegandolo, mediante un voltmetro, all'elettrodo
di \ce{Zn} immerso nella soluzione di \ce{ZnSO4}, abbiamo verificato che, una volta inserito il ponte salino, tra i due materiali si trovava una forza elettromotrice di 1.08 V $\pm$ 0.02 V (l'incertezza è stata assegnata notando di quanto variava il valore sul monitor del multimetro). Osservando il segno riportato dal multimetro abbiamo verificato che l'elettrodo di \ce{Cu} era l'anodo (+), mentre quello \ce{Zn} era il catodo (-).
Sia il valore ottenuto che il segno degli elettrodi è in accordo con i potenziali di Nerst, che prevedono
1.10 V per la coppia di materiali usati.

Sostituendo allo zinco l'elettrodo di argento, immerso nella soluzione di \ce{AgNO3} abbiamo misurato
una differenza di potenziale di 0.43 $\pm$ 0.01 V. In questo caso l'\ce{Ag} fungeva da anodo (+), mentre
il rame era il catodo (-).

Di nuovo, usando nelle rispettive soluzioni lo zinco e l'argento e procedendo in modo analogo,
abbiamo misurato la più alta differenza di potenziale pari a 1.50 $\pm$ 0.01 V,
con l'\ce{Ag} all'anodo (+) e lo \ce{Zn} come catodo (-).

Mettendo in serie rispettivamente 2 e 3 celle galvaniche di questo tipo, con i gruppi vicini,
abbiamo ottenuto: 2.94 $\pm$ 0.01 V e 4.40 $\pm$ 0.01 V. Con queste tensioni abbiamo provato ad alimentare un
diodo LED, che siamo riusciti a far funzionare, anche se emetteva pochissima luce. Evidentemente
le celle non sono in grado di fornire tensione o corrente a sufficienza. Il LED necessita di circa 100 mA
che si ottengono applicando 10 V ai suoi capi.

Per facilità di lettura riportiamo i dati nella Tabella \ref{tab:celle}. Siamo felici di notare che i dati concordano bene
con i potenziali standard, nonostante la temperatura fosse di 27$^\circ$C invece che 25$^\circ$C. 

\begin{SCtable}[1][h]
    \centering
    \begin{tabular}{l c c}
        \toprule
        \multicolumn{1}{c}{Cella} & \multicolumn{2}{c}{f.e.m. [V]} \\
        (anodo-catodo) & Nerst & Misurate \\
        \midrule
        Cu-Zn & 1.10 & 1.08 \\
        Ag-Cu & 0.46 & 0.43 \\
        Ag-Zn & 1.56 & 1.50 \\
        Ag-Zn (2x) & 3.12 & 2.94 \\
        Ag-Zn (3x) & 4.68 & 4.40 \\
        \bottomrule
    \end{tabular}
    \caption{Forze elettromotrici misurate con il multimetro. Nella prima colonna abbiamo riportato le coppie di metalli
    secondo la convezione anodo-catodo. I (2x) e (3x) indicano il numero di celle in serie. Le incertezze sono di 0.01 V.}
    \label{tab:celle}
\end{SCtable}

\subsection*{Misura della concentrazione di \ce{Ca^{2+}}}

Abbiamo misurato diverse acque, del latte e un succo di frutta, confrontando, ove possibile, i valori ottenuti con i dati riportati sull'etichetta. Nella seguente tabella sono esposti i risultati:

\begin{table}
\centering

\begin{tabular}{r @{\quad} c c c c c c}
\toprule
 & \multicolumn{4}{c}{acqua} & latte di vacca & succo di frutta \\
 & Guizza & Panna & Ferrarelle & di Piné & diluito & diluito \\
 \midrule
\phantom{.}[\ce{Ca2+}] (mg/l) & 14.3 & 33.4 & 460 & 14.8 & 6.34 & 10.6 \\
ddp (\si{\milli\volt}) & 15.1 & 25.2 & 56 & 15.5 & 5.6 & 11.6 \\
 \midrule
\phantom{.}[\ce{Ca2+}] etichetta & \multirow{2}{*}{10.6} & \multirow{2}{*}{32} & \multirow{2}{*}{392} & \multirow{2}{*}{} & \multirow{2}{*}{} & \multirow{2}{*}{} \\
(mg/l) &  &  &  &  &  &  \\
\bottomrule
\end{tabular}

\end{table}

Abbiamo misurato erroneamente, le concentrazioni di \ce{Ca2+} in soluzioni prive di ISA ottenendo risultati molto distanti da quelli nominali. Successivamente, una volta scoperto l'errore commesso, abbiamo preso delle nuove misure, le quali risultano molto vicine al valore nominale in nostro possesso.
Per l'acqua Guizza la quale riporta in etichetta una concentrazione di \ce{Ca2+} uguale a $10.6 \text{mg/l}$ mentre senza ISE otteniamo un valore di $26.8 \text{mg/l}$ con l'ISE si ottiene $14.3 \text{mg/l}$.
Analogamente con il latte, il quale è stato diluito 100 volte, abbiamo misurato senza ISA $6.34 \text{mg/l}$ mentre con si ha 10.5 mg/l. Il valore nominale, non reperibile sull'etichetta è stato ricercato nel web. Abbiamo trovato che un valore tipico è $1200 \text{mg/l}$ la quale, appunto, è molto vicina a quella misurata.

\section*{Conclusione}

Come già sottolineato in precedenza, abbiamo avuto dei problemi con la seconda regressione lineare.
Infatti l'andamento della conducibilità che abbiamo registrato non è del tutto lineare. La legge che descrive
la conducibilità in funzione della concentrazione, la legge di Kohlrausch, non è lineare ma ha l'andamento di una radice quadrata.
Occorre evidenziare che noi abbiamo variato la concentrazione in modo non lineare, poichè abbiamo aggiunto volume di
soluzione a concentrazione costante ad un altra soluzione. Questo fatto può spiegare come mai l'andamento è quasi lineare.

Nell'analisi dati abbiamo preferito conservare tutti i dati misurati e aggiustare le incertezze. Tuttavia un altro approccio
sarebbe possibile: rimuovere gli ultimi dati che costringono la retta a diventare più piana. Abbiamo quindi fatto un piccolo esperimento:
sono stati scartati tutti i dati al di sopra di 15 ml ed è stato eseguita una regressione. In questo caso il punto di inversione
è risultato molto vicino ai 6 ml. È stato comunque necessario aggiustare le incertezze, ma questa volta l'errore sulla conducibilità corretto
era di circa 0.09 mS, un valore molto più accettabile dei 0.3 mS del caso con tutti i dati. Con questo metodo la concentrazione incognita
ha assunto il valore di 0.091 M. 

Concludendo, possiamo affermare che il risultato ottenuto e in particolar modo la sua incertezza sono sottostimati. Possiamo però asserire con
sufficiente sicurezza che la concentrazione incognita è compresa tra 0.085 M e 0.095 M. 


\end{document}
