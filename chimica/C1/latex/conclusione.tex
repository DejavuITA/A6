\section*{Conclusione}

Come già sottolineato in precedenza, abbiamo avuto dei problemi con la seconda regressione lineare.
Infatti l'andamento della conducibilità che abbiamo registrato non è del tutto lineare. La legge che descrive
la conducibilità in funzione della concentrazione, la legge di Kohlrausch, non è lineare ma ha l'andamento di una radice quadrata.
Occorre evidenziare che noi abbiamo variato la concentrazione in modo non lineare, poichè abbiamo aggiunto volume di
soluzione a concentrazione costante ad un altra soluzione. Questo fatto può spiegare come mai l'andamento è quasi lineare.

Nell'analisi dati abbiamo preferito conservare tutti i dati misurati e aggiustare le incertezze. Tuttavia un altro approccio
sarebbe possibile: rimuovere gli ultimi dati che costringono la retta a diventare più piana. Abbiamo quindi fatto un piccolo esperimento:
sono stati scartati tutti i dati al di sopra di 15 ml ed è stato eseguita una regressione. In questo caso il punto di inversione
è risultato molto vicino ai 6 ml. È stato comunque necessario aggiustare le incertezze, ma questa volta l'errore sulla conducibilità corretto
era di circa 0.09 mS, un valore molto più accettabile dei 0.3 mS del caso con tutti i dati. Con questo metodo la concentrazione incognita
ha assunto il valore di 0.091 M. 

Concludendo, possiamo affermare che il risultato ottenuto e in particolar modo la sua incertezza sono sottostimati. Possiamo però asserire con
sufficiente sicurezza che la concentrazione incognita è compresa tra 0.085 M e 0.095 M. 
