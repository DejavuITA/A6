\section*{Materiale}

\begin{itemize}
        \setlength{\parskip}{-1pt}
        \item{Due becker, una buretta con relativo supporto, bacchetta di vetro, ``spatola'';}
        \item{Agitatore magnetico con ancoretta;}
        \item{Conduttimetro;}
        \item{Soluzioni di \ce{KCl} di concentrazione 0.1 M e 0.01 M per la taratura del conduttimetro,
                soluzione di \ce{AgNO3} di concentrazione da determinare, \ce{NaCl} e acqua;}
        \item{Agitatore magnetico con ancoretta;}
        \item{Francesco Pasa, Andrea Miani, Davide Bazzanella, possibilmente con un minimo di comprensione
                del da farsi.}
\end{itemize}

\section*{Procedura e incertezze}

Come prima cosa, abbiamo tarato il conduttimetro seguendo la procedura descritta nel manuale e utilizzando
le soluzioni di \ce{KCl} a nostra dispozisione. Abbiamo quindi pulito la cella conduttimetrica per evitare contaminazioni.

In seguito, abbiamo preparato 50 mL di soluzione titolante 1.5 M di \ce{NaCl}. Abbiamo considerato un
incertezza standard sul volume di 0.3 mL e sul peso di 0.003 g. Usando queste incertezze,
si ottiene che la soluzione titolante aveva una concentrazione $1.50 \pm 0.03$ M.

Con la stessa soluzione abbiamo lavato la buretta per evitare diluizioni non volute. La buretta è stata riempita con
la soluzione rimanente ed è stata posta sopra il becker contenente 100 mL di titolanda a concentrazione incognita.
Misurando ogni volta la concentrazione con il conduttimetro, la titolante è stata versata nella soluzione di \ce{AgNO3}
a passi di 0.5 mL. In questo modo avviene la reazione di precipitazione:

\begin{equation*}
        \ce{AgNO3\ped{(aq)} + NaCl\ped{(aq)} -> AgCl\ped{(s)} v + NaNO3\ped{(aq)}}
\end{equation*}

Poiché gli ioni \ce{Na^+} conducono meno dei \ce{Ag^+} la conducibilità diminuisce man mano che viene aggiunto NaCl,
finchè tutti gli Ag non sono precipitati. A questo punto stiamo aggiungendo nuovi ioni che aumentano la conducibilità,
per cui il trend si inverte. In questo modo si può risalire alla quantità di \ce{AgNO3} in soluzione e calcolare la concentrazione.

L'incertezza standard sul volume della soluzione titolanda (nel becker) è stata posta a 0.3 mL, mentre con la
buretta l'incertezza standard è di 0.03 mL, quindi l'incertezza sul volume versato, che è la differenza tra due volumi,
è di 0.04 mL.
