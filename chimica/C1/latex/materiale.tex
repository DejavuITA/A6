\section*{Materiale}

\begin{itemize}
        \setlength{\parskip}{-1pt}
        \item{Due becker, una buretta con relativo supporto, bacchetta di vetro, ``spatola'';}
        \item{Agitatore magnetico con ancoretta;}
        \item{Conduttimetro;}
        \item{Soluzioni di \ce{KCl} di concentrazione 0.1 M e 0.01 M per la taratura del conduttimetro,
                soluzione di \ce{AgNO3} di concentrazione da determinare, \ce{NaCl} e acqua;}
        \item{Agitatore magnetico con ancoretta;}
        \item{Francesco Pasa, Andrea Miani, Davide Bazzanella, possibilmente con un minimo di comprensione
                del da farsi.}
\end{itemize}

\section*{Procedura e incertezze}

Come prima cosa, abbiamo tarato il conduttimetro seguendo la procedura descritta nel manuale e utilizzando
le soluzioni 0.1 M e 0.01 M di \ce{KCl} a nostra dispozisione. Abbiamo quindi pulito la cella conduttimetrica per evitare contaminazioni.

In seguito, abbiamo preparato 50 mL di soluzione titolante 1.5 M di \ce{NaCl}. Abbiamo considerato un
incertezza standard\footnote{L'incertezza standard è uguale all'incertezza di risoluzione divisa per $\sqrt{3}$.} sul volume di 0.3 mL 
(si è ritenuto che il matraccio usato avesse un incertezza di risoluzione di 0.5 mL)
e sul peso di 0.003 g (era possibile leggere i centesimi di grammo mentre i millesimi fluttuavano, in questo caso abbiamo posto l'errore di risoluzione a 0.005 g).
Usando queste incertezze tipo, si ottiene che la soluzione titolante aveva una concentrazione $1.50 \pm 0.03$ M.

Abbiamo lavato la buretta con la soluzione per evitare diluizioni non volute. La buretta è stata riempita con
la soluzione rimanente ed è stata posta sopra il becker contenente 100 mL di soluzione titolanda a concentrazione incognita.
Misurando ogni volta la concentrazione con il conduttimetro, la titolante è stata versata nella soluzione di \ce{AgNO3}
a passi di 0.5 mL. In questo modo avviene la seguente reazione di precipitazione:

\begin{equation*}
        \ce{AgNO3\ped{(aq)} + NaCl\ped{(aq)} -> AgCl\ped{(s)} v + NaNO3\ped{(aq)}}
\end{equation*}

La parte interessante della reazione è che l'argento cloruro (\ce{AgCl}) precipita. Di conseguenza gli ioni \ce{Ag^+}
vengono rimossi dalla soluzione e non contribuiscono più alla sua conducibilità elettrica.
Poiché gli ioni \ce{Na^+} conducono meno dei \ce{Ag^+} la conducibilità diminuisce man mano che viene aggiunto NaCl,
finchè l'argento non è completamente precipitato. A questo punto, continuando ad aggiungere NaCl, si aggiungendo nuovi ioni \ce{Na^+} che fanno salire la conducibilità,
per cui il trend si inverte.

Conoscendo il volume della soluzione titolante necessaria per arrivare al punto di inversione, e quindi il numero di moli
di NaCl, si può risalire alla quantità di \ce{AgNO3} in soluzione e calcolare la concentrazione.

L'incertezza standard sul volume della soluzione titolanda (nel becker) è stata posta a 0.3 mL, mentre con la
buretta l'incertezza standard è di 0.03 mL. Per ottenere l'incertezza sul volume versato, che è la differenza tra due volumi,
occore comporre le incertezze. L'incertezza sul volume versato è quindi di 0.04 mL.
