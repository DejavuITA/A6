\section*{Scopo}

L'obiettivo di questa esperienza di laboratorio è quello di determinare la concentrazione
di una soluzione mediante misure di conduttimetria. Per fare ciò si sfrutta una reazione di
precipitazione, mediante la quale è possibile sostituire gli elettroliti della soluzione
a concentrazione incognita con altri ioni (nel nostro caso abbiamo sostituito \ce{Ag^+} con \ce{Na^+}),
che hanno una conducibilità diversa. In questo modo, misurando la conducibilità, si può
estrapolare la quantità di elettroliti originari e quindi la concentrazione.
