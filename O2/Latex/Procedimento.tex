\section{Procedimento \& Analisi dati}

\subsection{Relazione tra l'indice di rifrazione di un materiale e la lunghezza d'onda del raggio incidente}

La prima operazione eseguita è stata quella di misurare l'angolo ''zero''. Ovvero abbiamo misurato l'anglolo segnato dal goniometro quando tra la sorgente luminosa e il canocchiale, allineati, non vi era interposto alcun'oggetto. Ricordiamo che più si stringe la fenditura del collimatore più la misura dell'angolo ''zero'' risulta essere precisa. Questo compatibilmente con l'intensità luminosa osservabile da un occhio umano.
L'angolo ''zero'' quindi ha il seguente valore:

\begin{equation}
	\theta\ped{0} \,=\, 128.31^\circ \pm 0.01^\circ
\end{equation} 

Ora vogliomo osservare la relazione che sussiste tra l'indice di rifrazione del nostro prisma di vetro e la lunghezza d'ona di un raggio luminoso.
A tal fine occorre ricordare che il raggio luminoso incidente su una faccia di un prisma subisce una riflessione e una rifrazione, poichè passa da un mezzo ad un altro.
Consideriamo ora cosa accade al raggio rifratto. Poichè il raggio luminoso incidente non è monocromatico questo è composto da numerosi fasci luminosi con lunghezza d'onda differente, tutte quelle presenti nello spettro del visibile. Quindi quando il raggio incidente viaggia attraverso il prisma si scompone nelle sue varie componenti (dispersione della luce) a seconda delle diffrenti lunghezze d'onda dei fasci luminosi che lo comongono. Quindi i vari raggi luminosi subiscono un'ulteriore rifrazione passando nuovamente dal vetro all'aria, uscendo pertanto dal prisma. Questa seconda rifrazione non porta altro che ad un'amplificazione dell'effetto precedente, pertanto è più semplice distinguere i vari raggi luminosi.
In questa esperienza noi non siamo in grado di misurare direttamente la lunghezza d'onda di un raggio luminoso. Ciononostante sappiamo che ad un dato colore del raggio luminoso corrisponde una certa lunghezza d'onda, pertanto ora ci interessa trovare l'indice di rifrazione, del prisma, corrispondente ai nostri raggi luminosi monocromatici e quindi di una precisa lunghezza d'onda.
Noi sappiamo che, in un prisma, per ottenere l'indice di rifrazione di un materiale senza sapere la lunghezza d'onda del raggio uminoso incidente, basta conoscere l'angolo massimo ($\theta\ped{max}$) di uscita del raggio rifratto rispetto al proungamento della direzione incidente. Quindi noto questo valore applichiamo la sguente relazione:

\begin{equation}
	n \,=\, \frac{\sin{(\alpha\,+\,\frac{\theta\ped{max}}{2})}}{\sin{\frac{\alpha}{2}}}
\end{equation}

dove $\alpha$ è l'angolo di apertura del prisma, che nel nostro caso vale $60^\circ$, e ci ricaviamo così il valore dell'indice di rifrazione in base al colore (quindi indirettamente alla lunghezza d'onda) del raggio luminoso in esame.

Ora questa è la teoria che stà alla base dell'esperienza. Operativamente per misurare $\theta\ped{max}$ abbiamo adottato la seguente procedura:
