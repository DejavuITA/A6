\begin{table}[H]
    \centering
    \small
    \begin{tabular}{l c c}
        \toprule
        Colore & $\theta\ped(max) \pm d\theta\ped(max)$ [\si{\radiant}] & $n \pm dn$ \\
        \midrule
		Rosso 1	& 	$(6.717 \pm 0.004) \cdot 10^(-01)$ &	$1.5149 \pm 0.0003$ \\	
		Rosso 2	& 	$(6.722 \pm 0.004) \cdot 10^(-01)$ &	$1.5153 \pm 0.0003$ \\
		Rosso 3	& 	$(6.731 \pm 0.004) \cdot 10^(-01)$ &	$1.5159 \pm 0.0003$ \\
		Verde &		$(6.810 \pm 0.004) \cdot 10^(-01)$ &	$1.5210 \pm 0.0003$ \\
		Blu &		$(6.839 \pm 0.004) \cdot 10^(-01)$ &	$1.5229 \pm 0.0003$ \\
		Viola &		$(6.853 \pm 0.004) \cdot 10^(-01)$ &	$1.5238 \pm 0.0003$ \\
        \bottomrule
    \end{tabular}
    \caption{Dati relativi all'indice di rifrazioe $n$ in funzione del colore e quindi della lunghezza d'onda $\lambda$. sarebbe buona cosa mettere la lambda nella tabella}
    \label{tab:enne}
\end{table}
